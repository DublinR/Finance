\documentclass[]{article}

\begin{document}

\section{Ordinary Differential Equations}
%-----------------------------------------------------------------------------------%
%-----------------------------------------------------------------------------------%

\subsection{Euler integration method}
The Euler method is a first-order numerical procedure for solving ordinary differential equations (ODEs) with a given initial value. It is the most basic explicit method for numerical integration of ordinary differential equations and is the simplest Runge–Kutta method. 
%5.7 
\subsubsection*{Question}
Consider the following ODE
\[ \frac{df(t)}{dt}=tf(t),\]
with $f(t_0)=f_0$, and suppose you use Euler integration with time step $\Delta  T$, so that $t_n=t_0+n\Delta  T$. 
What is the value of the approximation at time $t_n$?

\begin{itemize}

\item[(a)] $f(t_n) \approx f_0 \prod^n_{k=1}(1+t_k\Delta  T)$

\item[(b)] $f(t_n) \approx f_0 \prod^{n−1}_{k=0}(1+t_k\Delta  T)$

\item[(c)] $f(t_n) \approx f_0+f_0\sum^{n−1}_{k=0}t_k\Delta  T$

\item[(d)] $f(t_n) \approx f_0+f_0\sum^n_{k=1}t_k \Delta  T$
\end{itemize}
%-----------------------------------------------------------------------------------%
%-----------------------------------------------------------------------------------%

\subsection{Reversion to the Mean}
% 5.8
Consider the following ODE for $\sigma(t)$,
\[  \frac{d\sigma(t)}{dt}= - 0.5(\sigma-0.20)\]

with $\sigma(0)=0.30$. How will $\sigma(t)$ evolve with time?

\begin{itemize}

\item[(a)] It will increase.


\item[(b)] It will oscillate around the value 0.30.


\item[(c)] It will decrease, reaching the value 0.20 in a finite time.


\item[(d)] It will decrease, approaching the value 0.20, but never reaching it.(correct)
\end{itemize}


\end{document}
