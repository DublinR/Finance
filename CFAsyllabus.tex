1. Ethics and professional issues <2>
2. Quantative methods <10>
3. Economics <16>
4. Financial statement analysis <14>
5. Corporate Finance <6>
6. Analysis of Equity investments <6>
7. Analysis of debt investments <14>
8. Evaluation of derivatives <5>
9. analysis of alternative investments <7>
10. portfolio management <16>

<96 subsections>
************************************
1. Ethical and professional standards
1.A professional standards of practice
1.B topical issues
	1.B.1	corporate governance
	1.B.2	soft dollar standards
	1.B.3 	Standards
	1.B.4   insider trading
		a. mosaic theory
		b. Selective Disclosure V Full Disclosure
	1.B.5	personal investing

*******************************************
2. Quantitative methods
2.A time value of money
2.B Basic Statistical concepts
2.c Probability concepts and random variables
2.D common  probability distributions
	2.D.1 Discrete Distributions
	2.D.2 Continuous Distributions
	2.D.3 Lognormal Distributionss.
2.E sampling and Estimation
2.F statistical inference and hypothesis testing
2.G Correlation analysis and linear regression
2.H multivariate regresssion 
2.I Time Series Analysis
	2.I.1 Trends
	2.I.2 limitations to trends
	2.I.3 fundamental issues in Time series analysis
	2.I.4 Autoregressive time series models
	2.I.5 random walks and unit roots
	2.I.6 Moving Average times series models
		Smoothing past values with a moving average
		moving average models for forecasting
	2.I.7 Seasonality in time series models
2.J portfolio concepts
	2.J.1 optimal portfolio with three assets
	2.J.2 minimum variance frontier for many assets
	2.J.3 instability in the minimum variance frontier
	2.J.4 Diversification and portfolio size
	2.J.5 risk free assets and the trade off between risk and return
	2.J.6 the Capital allocation line
	2.J.7 the Capital asset pricing model
	2.J.8 Estimates based on historical means
	2.J.9 the market model

***************************************
3 Economics
3.B Elasticity
3.C The Firm and Industry organization
3.C.1 Organization of the business firm
3.C.2 Costs of production 
3.C.3 Firms in competitive markets
3.E measuring national income
3.E.2 Components of GDP
3.E.3 Real v Nominal GDP
3.F Economic Fluctuation and unemployment
3.G The monetary System
3.K macro economics of an open economy
3.N Government regulation
3.L Aggregate Demand and Aggregate Supply
3.L.1 Aggregate demnd curve
3.L.2 Aggregate Supply curve
3.O natural resource markets
3.P relationship of economic activity to the investment process

4 Financial Statement Analysis
4.A financial reporting system
	General concepts and rules
	Generally accepted accounting procedures(GAAP)
	International accounting standards(IAS)
4.B principal financial statements
	balance sheet
	income statement
	Statemnet of cash flows
	statement of stockholders equity
4.C Earnings Quality and non-recurring items
4.D Analysis of Inventories
	relationship between inventories and costs of goods sold
	a-stable prices
	b-rising prices
	4.D.2 Inventory methods
		a- specific identifications
		b- First in First out
		c- Average cost
		d- Last in First out
4.E Analysis of long lived assets
	4.E.1 Capitalization v Expensing
		asset revaluation
	4.E.2 Depreciation
4.f Analysis of income taxes
4.G Analyis of financing liabilities
4.H Analysis of leases
4.H.1 Incentives for leasing
4.K Analysis of inter corporate investments
4.M Analysis of multinational Operations
4.M.2 Basic Accounting issues
4.N Ratio and financial analysis
4.N.5 profitability analysis
4.N.6 operation and financial leverage
4.N.7 earnings per share
****************************************************************
5 Corporate Finance
5.A Fundamental issues
	5.A.1 forms of business organization
	5.A.2 corporate governance issues
5.B Capital investment decisions	
	5.B.1 Investment Decision Critera
	5.B.2 cash flow projections
	5.B.3 project analysis and evaluation
5.C Business and financial risk
	5.C.1 Break even Analysis
	5.C.2 operating leverage
	5.C.3 Financial leverage
	5.C.4 Total combined leverage
5.D long term financial policy
	5.D.1 Cost of capital
5.E Mergers and acquisitions
	5.E.1
	5.E.2 Classifications
		vertical
		horizontal 
		Conglomerate
	5.E.3 Gains from acquisitions
	5.E.4 Defensive tactics
		supermajority clause in corporate charter
		Exclusionary Self tenders	
		posion pills

		going private andleveraged buyout
5.F Valuation implication of corporate finance
	5.F.1 capital investments decisions
	5.F.2 Long term financial policy
	5.F.3 mergers and acquisitions
************************************************************************
6 Analysis of Equity investments

6.A Organizaton and functioning of securities markets
	6.A.1 Primary Capital Markets
	6.A.2 Secondary financial markets
		A. Excahnge markets
		B. OTC markets
		C. Electronic Markets 
	6.A.3 Types of orders
6.B Security Market indices and benchmarks
6.C Equity Risk Definition (E.g. Statistical, economic, downside, relative, absolute, political and Measurement)
	6.C.1 Single Factor models
	6.C.2 Multi factor models
		A. Fundamental multifactor model
		B. Arbitrage Price Theory (APT)
		C. Practical limitations of risk meaurement for the equity analyst
		D. International Equity investing (e.g. emerging equity markets)	
6.D Fundamental Analysis	
	6.D.1 Theory of evaluation
*************************************************************************
7 Analysis of debt investments
7.A.1 Debt securities
7.B risk associated with investing in bonds
7.C GLobal bond sectors and instruments
7.D Yield Spread
7.E introduction to the valuation of securities
7.F yield measures,spot rates and forward rates.
7.G measurement of interest rate risk
************************************************************************************
8 Evaluation of derivatives
8.B Forward Markets and instruments
8.C Futures market
8.C.4 Application of futures
8.D The options market
8.D.3 Underlying instruments
8.D.3.A Bonds
8.D.4 Option trading
8.D.5 Valuing options
8.D.6 Option prcing (valuation ) model
8.D.6.A Binomial model
8.D.6.B Black Scholes Model
8.D.8 Option trading strategies
8.E Swaps markets
***********************************************************************************
9. Analysis of alternative investments

9.A real estate
9.B investment companies
	Valuing investment company shares
	closed end versus Open Ended Investment companies
	Fund management fees
	Investment Strategies
		Stule
		Sector
		Index
		GlobaL 
	 	Stable Value
	
9.C Venture Capital
	Stages of venture capital investing
	Risk
	Investment Characteristics
	Types of Liquidation
	Performance measurement	
9.D Hedge Funds
9.E closely held companies and inactively traded securities
	1. Legal Environments
	2. Valuation Alternatives
	3. Bases for discounts/ premiums 
9.F distressed securities and bankruptcies
9.G Commodity markets and commoditie derivatives
	1. Types of commodity difference
		Agricultural futures
		Energy futures
		Metals
	2. Fundamental concepts
	3. Analysis issues (e.g. Contract specification and delivery, cash and futures prices quotations)
	4. Spreads.
***********************************************************************************
10 portfolio management
10.A Capital Market Theory
	10.A.1 Markovitz Portfolio Theory
		A. Assumptions
		B. Inputs
		C. Implications
		D. Efficient Frontier
	10.A.2 Asset pricing models
10.B Management of Individual characteristics
	10.B.1 investor characteristics
		A. Life Cycle and Age influences
		B. Behavioral finance issues
	10.B.2 Objectives
		A. Establishing Return requirements
	10.B.3 Constraints
		A. Liquidity	
		B. Time Horizon
		C. Tax Exposure
		D. Legal and Regulatory
		E. Unique Circumstances
	10.B.4 Investment Policy statement
	10.B.5 Investment strategy and asset allocation
		A. Portfolio Construction
		B. Influences of taxes on investment strategy
10.C management of institutional investor portfolios
10.E Endowment fund and foundations
10.F Insurance companies
10.G Other corporate investors
10.I Asset allocation
10.j portfolio construction and revision
10.K Equity portfolio management strategies
10.M Real estate abd Alterntive invesgtment in portfolio management
10.O performance measurement
	sharpes ratio
	Treynor ratio
	jensens alpha
	information ratio
10.P Presentation of Performance results
