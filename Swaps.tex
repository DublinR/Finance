\documentclass[11pt]{article} % use larger type; default would be 10pt

\usepackage[utf8]{inputenc} % set input encoding (not needed with XeLaTeX)
\usepackage{geometry} % to change the page dimensions
\geometry{a4paper} % or letterpaper (US) or a5paper or....
\usepackage{graphicx} % support the \includegraphics command and options
\usepackage{booktabs} % for much better looking tables
\usepackage{array} % for better arrays (eg matrices) in maths
\usepackage{paralist} % very flexible & customisable lists (eg. enumerate/itemize, etc.)
\usepackage{verbatim} % adds environment for commenting out blocks of text & for better verbatim
\usepackage{subfig} % make it possible to include more than one captioned figure/table in a single float
% These packages are all incorporated in the memoir class to one degree or another...

%%% HEADERS & FOOTERS
\usepackage{fancyhdr} % This should be set AFTER setting up the page geometry
\pagestyle{fancy} % options: empty , plain , fancy
\renewcommand{\headrulewidth}{0pt} % customise the layout...
\lhead{}\chead{}\rhead{}
\lfoot{}\cfoot{\thepage}\rfoot{}

%%% SECTION TITLE APPEARANCE
\usepackage{sectsty}
\allsectionsfont{\sffamily\mdseries\upshape} % (See the fntguide.pdf for font help)
% (This matches ConTeXt defaults)

%%% ToC (table of contents) APPEARANCE
\usepackage[nottoc,notlof,notlot]{tocbibind} % Put the bibliography in the ToC
\usepackage[titles,subfigure]{tocloft} % Alter the style of the Table of Contents
\renewcommand{\cftsecfont}{\rmfamily\mdseries\upshape}
\renewcommand{\cftsecpagefont}{\rmfamily\mdseries\upshape} % No bold!

\begin{document}
\tableofcontents

\section{ Swaps}

A swap is one of the most simple and successful forms of OTC-traded derivatives. It is a cash-settled contract between two parties to exchange (or "swap") cash flow streams. As long as the present value of the streams is equal, swaps can entail almost any type of future cash flow. They are most often used to change the character of an asset or liability without actually having to liquidate that asset or liability. For example, an investor holding common stock can exchange the returns from that investment for lower risk fixed income cash flows - without having to liquidate his equity position. 
 
%-----------------------------------------------------------------------------------------------%

\bigskip
\noindent \textbf{The difference between a forward contract and a swap is that a swap involves a series of payments in the future, whereas a forward has a single future payment.}


\section{Types of Swaps}
Two of the most basic swaps are:

\begin{description}
\item[Interest Rate Swap] - This is a contract to exchange cash flow streams that might be associated with some fixed income obligations. The most popular interest rate swaps are fixed-for-floating swaps, under which cash flows of a fixed rate loan are exchanged for those of a floating rate loan. 
\item[Currency Swap] - This is similar to an interest rate swap except that the cash flows are in different currencies. Currency swaps can be used to exploit inefficiencies in international debt markets.
\end{description}

For example, assume that a corporation needs to borrow $\$10$ million euros and the best rate it can negotiate is a fixed $6.7\%$. In the U.S., lenders are offering $6.45\%$ on a comparable loan. The corporation could take the U.S. loan and then find a third party willing to swap it into an equivalent euro loan. By doing so, the firm would obtain its euros at more favorable terms. Cash flow streams are often structured so that payments are synchronized, or occur on the same dates. This allows cash flows to be netted against each other (so long as the cash flows are in the same currency). Typically, the principal (or notional) amounts of the loans are netted to zero and the periodic interest payments are scheduled to occur on that same dates so they can also be netted against one another.


%------------------------------------------------------------------------------------------------%
\end{document}
