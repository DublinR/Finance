

Valuation of Securities
Black Scholes Derivation

Risk Neutral Probability
Risk Neutral Measures
Multiperiod Models
Arbitrage opportunity

An opportunity is a predictable self-financing process H such that

V0(H) =iHi(1)Si(0) = 0

VT(H,) =iHi(T,)Si(T,)0		w

Binomial Model
Valuation of Securities
Black Scholes Derivation



The first step in deriving the Black-Scholes equation is to construct a replicating portfolio, which 
consists of a risk-free bond and stock to mimic the payoffs of the given derivative. Before 
moving on, we will describe the assumptions we use to derive the Black-Scholes equation. 
(2.1) 
(2.2) 4
Assumptions
• The stock price follows a geometric Brownian motion process with    and  σ constant. 
• The short selling of securities with the full use of proceeds is permitted. 
• There are no transactions costs or taxes. All securities are perfectly divisible. 
• There are no dividends during the life of the derivative. 
• There are no risk-free arbitrage opportunities. 
• Security trading is continuous. 
• The risk-free rate of interest, r, is constant and the same for all maturities.

http://www.stanford.edu/~japrimbs/Publications/OnBlackScholesEq.pdf


Section 10

Perpetual Option - XPO

For investors, perpetual options represent the highest ratio of possible risk/reward payoff compared to existing financial products. Perpetual options are viewed as “plain vanilla” options. For many investors they represent an advantage over other instruments (where dividends and/or voting rights are not a high priority) because the strike price on a perpetual option enables the holder to choose the buy or sell price point instead of having to select a singular stock price. In addition, XPOs can be preferable to standard options because they eliminate the expiration risk.

S is the current price of the underlying asset
K is the exercise price
r is the risk free interest rate
q is the dividend yield
 is the volatility of the underlying asset

Cp=Kh1-1h1- 1h1SKh1

h1=12-r-q2+r-q2-122+2r2
 
