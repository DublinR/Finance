http://www.casact.org/pubs/forum/03wforum/03wf621.pdf
What is credibility?
Credibility theory is all about weighted averages. Different estimates of a quantity are to be
weighted together. The more credible estimates get more weight.

In the context of esrimating expected losses for a member of a class, there are two natural estimates:
the experience of the member itsdf, and the average of the entire class. 

The former is more relevant but also more volatile than the latter. Two general approaches have been taken to
calculating weights in this case. The limited fluctuation approach is willing to accept the member
experience at face value if it meets a pre-defined standard of stability (full credibility) and if not
reduces the weight enough for the weighted average to meet the stability requirement. 

The greatest accuracy approach measures relevance as well as stability and looks for the weights that
will minimize an error measure. The average of the entire dass could be a very stable quantity,
but if the members of the class tend to be quite different from each other, it could be of less
relevance for any particular class. So the relevance of a wider class average to any member's
mean is inversely related to the variability among the members of the class.

The error measure used in the greatest accuracy approach is almost always expected squared error,
so this method is often called "least squares credibility." In Europe it is sometimes called
"classical credibility." The limited fluctuation approach is called classical in North America. Thus
"classical" is a term worth avoiding, not only because of its geographic ambiguity, but also because
it is a historical rather than a methodological description.

Least squares credibility

Suppose you have two independent estimates x and y of a quantity, with respective expected
squared errors u and v. Take a weighted average a = zx + (1-z)y. The expected squared error of
623
a is w = zZu + (l-z)%. 
What z minimizes w? Here is where the calculus comes in. The derivative dw/dz is 2zu + 2(z-1)v. 
If you set that to zero you get: zu+zv = v, or z = v/(u+v). Then
1-z =u(u+v). 
This makes it look like each estimate gets a weight proportional to the expected
squared error of the other. To express the weights as properties of the estimates themselves, note that

  (1/u)(1/u) + (l/v)=11+(u/v)=vu+v= z.

This shows that each estimate gets a weight proportional to the reciprocal of its expected squared error 1. Least squares credibility is
an application of this principle.

As an example, consider a class of risks. Suppose the losses I~i in year j for the ith member of the
class are randomly distributed as follows:
LIj ~- C + M i "4" •ij (1)

where C is the class mean loss, C+Mi is the mean loss for the i-th member, and ~i is the random
component for the jth period for this member. 

It is not much of a restriction to assume that the Mi's average to zero as do the I~,i's. 

Suppose the variance of the ivy's is t z and the variance of the random components I~ii all are si 2. Denote their average E(si 2) by s 2.

Somerimes t2 is called the variance of the hypothetical means and s 2 the expected process variance.
"Hypothetical" refers to the fact that the means C+Mi are not observed.
With this setup, consider two estimates of member i mean losses: x, the average losses of the
member for n periods, and y, the class mean loss C, which for now we will assume to know or at
least be able to estimate well enough to ignore the error. To apply the inverse variance weightings,
we needed to know the expected squared errors of x and y from the true value of C + M i.

By the definitions, y's expected squared error is just t z. The expected squared error ofx is the
expected value of its variance siZ/n, i.e., s2/n. 

Then applying the inverse expected squared error principle gives a weight to x of z;

	z =(n/s2)(n/s2)+(1/t2)=nn +(s2/t2). 

This is the original
Buhlmann credibility formula.
