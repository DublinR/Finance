Islamic Finance
 
Islamic Finance Qualification
4.5 Mudarabah (Profit Loss Sharing)
Mudarabah is an arrangement or agreement between a capital provider and an entrepreneur, whereby the entrepreneur can mobilize funds for its business activity. Any profits made will be shared between the capital provider and the entrepreneur according to an agreed ratio, where both parties share in profits and only capital provider bears all the losses if occurred. The profit-sharing continues until the loan is repaid. The bank is compensated for the time value of its money in the form of a floating interest rate that is pegged to the debtor's profits.

4.8 Musharakah (Joint Venture)
This concept is normally applied for business partnerships or joint ventures. The profits made are shared on an agreed ratio, while losses incurred will be divided based on the equity participation ratio. This concept is distinct from fixed-income investing (i.e. issuance of loans).
 
4.10 MURABAHA
Literally, murabaha means a sale on mutually agreed profit. Technically, it is a contract of sale in which the seller declares his cost and profit. Islamic banks have adopted this as a mode of financing. As a financing technique, it involves a request by the client to the bank to purchase certain goods for him. The bank does that for a definite profit is stipulated in advance.
 
4.18 Ijara
IJARAH
Ijarah is a contract of a known and proposed usufruct against a specified and lawful return or consideration for the service or return for the benefit proposed to be taken, or for the effort or work proposed to be expended. In other words, Ijarah or leasing is the transfer of usufruct for a consideration that is rent in case of hiring of assets or things and wage in case of hiring of persons.
IJARAH-WAL-IQTINA
A contract under which an Islamic bank provides equipment, building, or other assets to the client against an agreed rental together with a unilateral undertaking by the bank or the client that at the end of the lease period, the ownership in the asset would be transferred to the lessee. The undertaking or the promise does not become an integral part of the lease contract to make it conditional. The rentals as well as the purchase price are fixed in such manner that the bank gets back its principal sum alongwith with profit over the period of lease.
 
4.21 BAI SALAM
Bai salam means a contract in which advance payment is made for goods to be delivered later on. The seller undertakes to supply some specific goods to the buyer at a future date in exchange of an advance price fully paid at the time of contract. It is necessary that the quality of the commodity intended to be purchased is fully specified leaving no ambiguity leading to dispute. The objects of this sale are goods and cannot be gold, silver, or currencies. Barring this, Bai Salam covers almost everything, which is capable of being definitely described as to quantity, quality, and workmanship

4.24 ISTISNA'A
Istisna'a is a contractual agreement for manufacturing goods and commodities, allowing cash payment in advance and future delivery or a future payment, and future delivery. Istisna’a can be used for providing the facility of financing the manufacture or construction of houses, plants, projects, and building of bridges, roads, and highways.
 
