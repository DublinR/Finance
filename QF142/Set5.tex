

ARCH in mean Model
NARCH
SQARCH
Stochastic Volatility Modela
Bollerslev
Riskmetric Exponential Smoother

Models with a Leverage Effect

%============================================================================================%
\subsection*{Black's Observation}
Black (1976) was perhaps the first to observe that stock returns are negatively correlated with changes in volatility: that is, volatility tends to rise (or rise more) following bad news (a negative return) and fall (or rise less) following good news (a positive return). 

This is called the leverage effect, as firms’ use of leverage can provide an explanation for this correlation: if a firm uses both debt and equity then as the stock price of the firm falls its debt-to-equity ratio rises. This will raise equity return volatility if the firm’s cash flows are constant. The leverage effect has since been shown to provide only a partial explanation to observed correlation, but the name persists.


ARCH in mean Model
NARCH

%============================================================================================%
\subsection*{SQARCH}

Stochastic Volatility Models
Bollerslev 1990
Bollerslev’s model is both parsimonious and ensures positive definiteness. However, there exists some empirical evidence against the assumption that conditional correlations are constant.
Nevertheless, Bollerslev’s model is a reasonable benchmark against which to compare any alternative multivariate GARCH model.

(a) Describe the constant conditional correlation model proposed by Bollerslev, 1990.
Riskmetric Exponential Smoother

(b) Describe the RiskMetrics exponential smoother model for multivariate volatility.
