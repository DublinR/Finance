
\section{MetallGesellSchaft}

- Losses made by MG's energy group, to the value of 1.5 Billion USD in December 1993
- These losses resulted in huge cash flow problems resulting from Large Oil Contracts that it had written

learning outcomes
 - trading strategies employed by the company
 - how proper supervision could have averted disaster
 - how similar crises may be averted in the future
 - events leading up to the losses, the risks incurred and the mitigation process described

Potential mitigation
MGs managements would have benefitted from implementing the recommendations put forward by G30.

\subsection{MG}

\begin{itemize}
\item MG Hedged Oil Contracts that they sold. They expired years in the future, so they decided to hedge with short term risk, and roll them over
\item They hedged the same number of barrels, this exposed them to enormous amounts of basis risk.
\end{itemize}


\newpage
MetalGesellschaft


In 1993, the company lost 1.3 billion dollars after speculating increase in oil price in oil futures market. A subsequent drop in oil price left the company buying the oil at a higher price than the market price. 
 



Company Profile

Metallgesellschaft Corporation (MG) is the subsidiary of Metallgesellschaft A.G., a German conglomerate with 15 major subsidiaries closely held with over 65% of stock owned by
institutional investors including banks.  In 1993, MG’s trading subsidiary, MG Refining and Marketing (MGRM), established very large energy derivatives  (futures and swaps) positions to
hedge its price exposure on its forward-supply contracts to deliver gasoline, diesel fuel and heating oil (about 160 million barrels) to its customers over a period of ten years at fixed prices. The counter-parties to forward contracts were retail gasoline suppliers, large manufacturing firms, and some government entities.  The central premise of their forward contracts is to supply oil at fixed price to independent retailers who often face severe liquidity crisis and squeezes on margin when oil prices rise.  It believed it is possible to arbitrage between the spot oil market and the long-term contract market.  This arbitrage required skilled use of the futures markets in oil products, and this was to be MGRM’s stock in trade.

THE HEDGING STIUATION

MGRM developed several novel contract programs.  First, it offered a “firm-fixed” program under which the customer would agree to a fixed monthly delivery of oil products at a set price.  (102 million barrels of oil products were obligated under this program by September 1993).  Second, it offered a “firm-flexible” contracts under which the customers were given
extensive rights to set the delivery schedule for up to 20% of its needs in any year, besides the fixed price commitments.  (A total of 52 million barrels were contracted under this program).

Third, it offered a “guaranteed margin” contracts under which it agreed to make deliveries at a price that would assure the independent operator a fixed margin relative to the retail price offered by its geographical competitors.  The contracts could be extended annually for a defined period and at MGRM’s discretion.  This means they were not firm obligations.  (By September 1993, a total of 54 million barrels were committed under this program.)  It is the first two programs involving 154 million barrels of obligations for periods up to ten years that constituted MGRM’s designated short position in oil. Most of the forward contracts were negotiated during the summer of 1993 when energy prices were low and falling and the contracts came with cash-out option if the energy price were to rise above the contractually fixed prices.  The fixed delivery prices were set 3 to 5 dollars higher than the spot price when writing the contracts.  Under the cash-out provision, the buyer could choose to sell the remainder of its forward obligations back to MGRM for a cash payment of one-half the difference between the prevailing near month futures price and the contractually fixed supply price times the total volume remaining on the contract.  MGRM opted for early exercise sell-back options instead of negotiated unwinding. These options take effect when the front-month futures rises above the fixed delivery price in the flow contract.   Although customers might wish to exercise these sell-back options, if they expect spot prices in the future to fall, they might well wish to do so even if they regarded a surge in spot prices as permanent. Remember, that they must compare the immediate cash payment with the PV of expected future difference between spot prices and the delivery prices over the remaining life of the contract.

EXPOSURE 
MGRM’s fixed price forward delivery contracts exposed it to the risk of rising energy prices.  MGRM hedged this price risk with energy futures contracts of between one to three
months to maturity at NYMEX and OTC swaps. The objective of its hedging strategy was to protect the profit margins in its forward delivery contracts by insulating them from increases in
energy prices.  MGRM would gain substantially from its derivative positions if the energy prices rise.  During the later part of 1993, however, energy prices fell sharply ($19 a barrel in June 93 to $15 a barrel in Dec. 93) resulting in unrealized losses and margin calls on derivative positions in excess of $900 million.  To complicate the matter, the futures market went into a contango price relationship for almost entire year in 1993 increasing cost each time it rolled its derivatives.  

The MG’s Supervisory Board responded to the situation of mounting margin calls by replacing MG’s top management and liquidating MGRM’s derivative positions and forward supply contracts which ended MG’s involvement in the oil market.  It suffered derivative related loss of $1.3 billion by the end of 1993. The new management team declared that “speculative oil deals…..had plunged MG into the crisis…..”  Only a massive $1.9 billion rescue operation by 150 German and international banks kept MG from going into bankruptcy. One reason for not buying forward contracts for the same maturity is that market for long-dated oil contracts is small - only about 10 firms made prices in this market. Another reason is that the MG’s credit rating was low enough for those firms to be exposed to it for long. (Economist)  If energy prices had risen rather than fallen, MGRM would not have had a problem.  It would have had unrealized gains on its derivatives position, and positive margin flows from the forward contracts.  Although it would have had unrealized losses on its forward contracts, it would not have mattered as it would be offset by the unrealized gains on its derivative positions.
MGRM’s hedging strategy 
MGRM’s hedging strategy included short-dated energy futures contracts and OTC swaps – a “stack and roll” or “rolling stack” strategy.  Under this strategy, MGRM opened a long position in futures staked in the near month contract.  Each month MGRM would roll the stack over into the next near month contract, gradually decreasing the size of the position.  Under this plan the total long position in the stack would always match the short position remaining due under the supply contracts.  It bought long futures positions on the NYMEX (equivalent to 55 m. barrels of gasoline, heating oil and crude oil) and entered into OTC energy swaps (100 to 110 m. barrels) with swap dealers (mostly banks) entitling it to receive payments based upon floating energy prices while making fixed payments.  

MGRM’s total derivative’s position was almost equal to its forward comitments, a barrel for barrel hedge or with hedge ratio of one.  The short- term nature of the derivatives called for continuous roll forward to maintain the hedge position.  This exposed the firm to rollover risk.  A stack hedge refers to a futures position being “stacked” or concentrated in a particular delivery month (or months) rather than being spread over many delivery months.  The stack and roll strategy can be profitable when markets are in “backwardation,” that is, when spot prices are higher than futures prices.  But when markets are in “contango,” that is, when futures prices are higher than spot prices, the strategy will result in losses. Critics assert that MGRM’s strategy exposed it to three significant and related risks: rollover risk, funding risk, and credit risk, because of the maturity mismatch between the hedge and the delivery contracts and other features.  It was exposed to rollover risk because of uncertainty about whether it would sustain gains or losses when rolling its derivatives position forward.  It is exposed to funding risk because of the marked-to-market conventions that applied to its short-dated derivative’s position.  It is exposed to credit risk because of its forward delivery counter-parties might default on their long-dated obligations to purchase oil at fixed prices.  If the energy prices fell, this risk is expected to increase because of the increase in the difference between contractual prices and prevailing spot prices.   To minimize the credit risk, MGRM limited the annual volume supplied under contract to no more than 20% of the customer’s needs and included in the contracts a cash-out option.  It could, however, be a factor in MGRM’s ability to raise funds against the collateral of these contracts.

Another feature of MGRM’s hedging strategy, which entails mismatched maturity structure, is that it exposed the firm to excessive amount of basis risk-variations in the value of
the short-dated futures positions not compensated by equal and opposite variations in the value of the long-dated delivery contracts because of  a one-for-one hedge it entailed.  One barrel of oil for delivery in one month is simply not equal in PV to one barrel of oil for delivery in ten years and the value of two different dated obligations do not move in lock step.  In general, spot prices are more variable than the futures prices.  This is a feature that all hedgers must deal with. Hedgers in the futures market are “speculators on the basis,” trading greater price risk for a lessor basis risk.  The basis risk is the difference between the price of the instrument and the price of the underlying asset being hedged.

A rolling stack of short-dated futures initially increases the variance of cash flows.  This occurs because movements in the price of oil within the month create losses or gains on the
entire stack of contracts.  These losses or gains must be settled by the end of the month; while compensating gains or losses on deliveries are realized only gradually over the remaining ten years of the delivery contract.  When cash flows matter, the rolling stack may be worse than no hedge at all.
CONCLUSION
MGRM has been losing money on its futures position throughout 1993.  The consequences had already been felt within the U.S subsidiary by the end of the summer as the
firm’s credit lines were used up.  When the oil price fell yet more precipitously at the end of the year, the company did not have sufficient cash to continue to roll over its stack of oil futures contracts as planned and could not meet a large number of its other obligations until it received an emergency line of credit from its bankers.  Losses eventually totaled nearly $1.3 billion.  By January the firm was close to declaring bankruptcy and its future was not clear. MG eventually negotiated a $1.9 billion bailout from its bankers in tandem with a plan to
shed assets such as its auto parts manufacturing business, its tin mining operations, its recently acquired heating equipment and others.  The price of MG share fell by half between November 1993 and February 1994 as a consequence.
