\documentclass[PRMIA4A.tex]{subfiles} 

\begin{document} 

%---------------------------------------------------%
\newpage
\section{BANKERS TRUST}

This case study focuses on the losses and loss of reputation at Bankers Trust (BT) in 1994 after it was sued by four of its major clients who asserted that Bankers Trust had misled them with respect to the riskiness and value of derivatives that they had purchased from the bank.

\subsection{Learning Outcomes}
The candidate should be able to:
\begin{itemize}
	\item Describe the Timeline of Events
	\item Describe the lessons learnt
	\item Discuss the events leading up to the losses, the risks incurred
	and the mitigation processes described
\end{itemize}
%================================================================================================= %
\subsection{Gibson Greetings }

Gibson Greetings was a greeting card and wrapping paper manufacturing company. It was conservative company which was naïve to derivatives. It did not want to incur loss more than $\$3$ million through derivative speculations. Over eight months in derivative contracts with Bankers’ Trust it earned $\$ 260000$. This profit made Gibson comfortable about derivatives. After that the company entered into around 29 linked derivatives to earn more profit and paid to Bankers Trust  around $\$13$ million. 

These complex derivatives had fancy names like \textbf{\textit{ratio swap, periodic floor,  spread lock 1 and 2}}, Treasury-linked swap, knockout call option, Libor-linked payout, time swap, and wedding band 3 and 6. 


As many of the contracts contained options, they incorporated  leverage (having fixed cost like loan). But after some time Gibson started losing money which  was much more than the maximum loss amount specified by Gibson. Its losses increased dramatically in response to small changes in interest rates due to high leverage. When Gibson suffered from $\$17.5$ million loss, Bankers Trust made it enter into another contract that could  lead to reduced loss of $\$3$ million or increased loss of $\$27.5$ million. This bet also failed and increased the loss to $\$20.7$ million After Gibson Greetings lost huge amount, it sued Bankers’ Trust.

\subsection{Summary}
\begin{itemize}
	\item BT’s reputation took a pounding after the bank was sued by several customers alleging various forms of fraud
	and racketeering with respect to derivatives transactions they had entered into with the bank. Several of these
	suits have since been settled both in and out of court, costing the company millions of dollars in settlement and
	possibly much more in damage to its reputation.
	\item The root cause appears to have been that BT’s clients felt that BT had unfairly exploited their comparative lack of
	sophistication in handling these sophisticated derivative products.
	\item This appears to be an example of poor stakeholder management. In focusing on increasing profits, Bankers Trust
	didn’t pay adequate attention to the fact that its clients were vital to its business. Even if it did nothing dishonest,
	it failed to serve its clients in terms of making them feel informed and at ease with their deals.
\end{itemize}
%---------------------------------------------------%

%------------------------------------------------%
%------------------------------------------------%
% Banker's Trust

Banker's Trust was sued by 4 of it's major clients

\begin{enumerate}
\item Federal Paper Board Company
\item Gibson Greeting
\item Air Products and Chemicals
\item Proctor and Gamble
\end{enumerate}

The charge was that BT misled them in respect to the riskiness and value of derivatives.



%------------------------------------------------%
%------------------------------------------------%

\subsection{Banker's Trust}


Bankers Trust became a leader in the nascent derivatives business under the management of Charlie Sanford, who succeeded Alfred Brittain III, in the early 1990s. Having de-emphasized traditional loans in favor of trading, the bank became an acknowledged leader in risk management. Lacking the boardroom contacts of its larger rivals, notably J. P. Morgan, BT attempted to make a virtue of necessity by specializing in trading and in product innovation.

In early 1994, despite all its prowess in managing the risks in the trading room, the bank suffered irreparable reputational damage when some complex derivative transactions caused large losses for major corporate clients. Two of these -- Gibson Greetings and Procter \& Gamble (P\&G) -- successfully sued BT, asserting that they had not been informed of, or (in the latter case), had been unable to understand the risks involved.
 
%------------------------------------------------%
%------------------------------------------------%
\subsection{Events}
BT’s reputation took a pounding after the bank was sued by several customers alleging various forms of fraud and racketeering with respect to derivatives transactions they had entered into with the bank. Several of these suits have since been settled both in and out of court, costing the company millions of dollars in settlement and
possibly much more in damage to its reputation.

The root cause appears to have been that BT’s clients felt that BT had unfairly exploited their comparative lack of sophistication in handling these sophisticated derivative products.

This appears to be an example of poor stakeholder management. In focusing on increasing profits, Bankers Trust didn’t pay adequate attention to the fact that its clients were vital to its business. Even if it did nothing dishonest, it failed to serve its clients in terms of making them feel informed and at ease with their deals.
 

On October 27, 1994, Procter \& Gamble filed suit against Bankers Trust for fraud and racketeering in transactions in equity swaps that had produced a one hundred million dollar loss.
Their suit laid bare the moral hazards associated with OTC derivative trading. Banks with proprietary trading desks in derivatives have a natural conflict of interests with clients.
The P\&G suit lay bare the moral hazards in OTC derivative trading.

Traders inevitably end up with positions they don't want and resort to high pressure selling to unload these positions on trusting clients so as not to miss performance targets.
This gives clients who are not financial intermediaries, like Procter \& Gamble, an opening to allege fraud when their bets turn sour.

Bankers Trust, a pioneer in the OTC derivatives market, survived the financial cost of the P\&G suit, but later succumbed to the same market pressures that downed LTCM following the Russian bond default in 1998.

Bankers Trust no longer exists, having been swallowed by Deutsche Bank.
%------------------------------------------------%
%------------------------------------------------%

\subsection{Background of Bankers Trust}
The Bankers Trust, a famous American Financial Institute was originally set up by banks which  could not perform trust services. A consortium of banks all invested in a new trust company and  thus formed Bankers Trust. The idea was a new trust company would not try and poach their  existing customer. Bankers Trust became a big name in the nascent derivatives business in the  early 1990s under leadership of Charlie Sanford. In comparison of JP Morgan’s strength of Board room contacts, Bankers’ Trust focused on specialization in trading and product innovation.
Bankers Trust

\begin{itemize}
\item Involved in specialized trading in derviatives
\item Sued by Gibson Greeting Cards and Proctor and Gamble (P&G), who asserted that Bankers Trust
misled them about the risks involved, and the value of the derivatives purchased from BT.
\item This was national news in 1994 in the USA
\item BT unfairly exploited their clients lack of sophistication in derivatives.
\item Several BT bankers were caught on tape remarking that the Gibson Greeting company would not
be able to understand what they were doing.
\item CEO Frank Newman led the company on an aggressive lending campaign
\item The bank suffered major losses in summer 1998 when Russia's default occurred.
\item The company was sued by
\begin{enumerate}
\item Federal Paper Board Company
\item Gibson Greeting 
\item Air Product and Chemicals
\end{enumerate}
The three cases were settled out of court for \$93 million.
\item The case with P\&G was not settled out of court.
\item P\&G entered into a transaction which pre-supposed that interest rates would remain stable or fall. In fact, interest rates rose.
\item The nature of the transaction with P\&G was a complex interest rate derivative transaction.
\item The net gain for P\&G was \$78 million.
\item This case is an example of Operational Risk - poor sales practices.
\end{itemize}

%=========================================================================================%
\end{document}



