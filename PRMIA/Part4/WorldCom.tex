PRMIA WorldCom

worldcom



WorldCom
Mergers and Acquisitions
When was the fraud discovered?
How was the fraud committed?
Bernie Ebbers


WorldCom
For a time, WorldCom was the United States's second largest long distance phone company (after AT&T). WorldCom grew largely by aggressively acquiring other telecommunications companies, most notably MCI Communications. 

WorldCom also owned the Tier 1 ISP UUNET, a major part of the Internet backbone.


In June 2002, Securities and Exchange Commission (SEC) lawyers filed civil fraud charges against WorldCom for what would later be estimated at over $9 billion worth of accounting errors.



Mergers and Acquisitions

MCI acquisition

On November 10, 1997, WorldCom and MCI Communications announced their US$37 billion merger to form MCI WorldCom, making it the largest merger in US history. On September 15, 1998 the new company, MCI WorldCom, opened for business.

Proposed Sprint merger

On October 5, 1999 Sprint Corporation and MCI WorldCom announced a $129 billion merger agreement between the two companies. Had the deal been completed, it would have been the largest corporate merger in history, ultimately putting MCI WorldCom ahead of AT&T as the largest communications company in the United States. However, the deal did not go through because of pressure from the US Department of Justice and the European Union on concerns of it creating a monopoly. On July 13, 2000, the boards of directors of both companies acted to terminate the merger. Later that year, MCI WorldCom renamed itself to simply "WorldCom" without Sprint being part of the company.



When was the fraud discovered?
While the current suit wasn’t filed until June 2002, it is apparent that some in the public were aware of illegal practices at WorldCom over a year earlier. A previous lawsuit was filed in June 2001 by several WorldCom shareholders, only to be thrown out. That suit included testimony from a dozen former WorldCom employees, detailing the same problems that would eventually bring about the company’s downfall.

The fraud was accomplished primarily in two ways:
1) Underreporting ‘line costs’ (interconnection expenses with other telecommunication companies) by capitalizing these costs on the balance sheet rather than properly expensing them.
2) Inflating revenues with bogus accounting entries from "corporate unallocated revenue accounts".





How was the fraud committed?
Financial executives at WorldCom exercised various methods of hiding expenses for a period of more than two years between 2000 and 2002. They delayed reporting some expenses and misrepresented others to give investors the appearance of growth during secretly hard times.
 

Bernie Ebbers
Ebbers was convicted of overseeing the $11 billion WorldCom fraud — much of it a pattern of chalking up expenses as long-term capital expenditures, which are classified as assets.


%-----------------------------------------------------------------------------------%


Jack Grubman
SEC fiend him with $15 million and banned him from Security Transactions for life.
He told Business week that he seemed to mock the ethicasl norms against conflict of Interest.
Grubam certainly rewarded executives for their close relationship with them.


%----------------------------------------------------------------------------------------------------%
Liberal Interpretation of Accounting Rules when preparing statements
it made it look like there was significant losses early on, but that the position then improved.

Accounting receivables
WorldCom also manipulated their A/C receivables.
