\documentclass[PRMIA4A.tex]{subfiles} 

\begin{document} 
	
	%---------------------------------------------------%
	\newpage

\newpage
\section{BANKGESELLSCHAFT BERLIN}

This case study focuses on the losses incurred at Bankgesellschaft Berlin, one of Germany’s 10 largest banks in summer 2001.

\subsection{Learning Outcomes}
The candidate should be able to:
\begin{itemize}
	\item Describe the Timeline of Events
	\item Describe how property-based funds carried unforeseen and uncovered risks
	\item Describe the lessons learnt
	\item Discuss the events leading up to the losses, the risks incurred and the mitigation processes described.
\end{itemize}


\subsection{Summary}
\begin{itemize}
	\item BgB sets up property-backed funds that are a lot riskier for the bank than for its retail investors. It also
	approves a series of risky loans to property developers and becomes involved with many of Berlin’s landmark
	regeneration projects.
	\item The property bubble bursts in Berlin and surrounding regions leading to massive losses and liabilities in the
	bank’s property-linked portfolios.
	\item Early in summer 2001, the Berlin senate was informed that Bankgesellschaft Berlin, one of Germany’s 10 largest
	banks, needed an emergency transfusion of E2 billion in new capital.
\end{itemize}



%Property Deals

%http://www.esmt.org/en/263575


\subsection*{Connections between Politicians and Business}

Rarely has the close intertwining of business and politics, with the accompanying unrestrained enrichment of party officials and their political favourites, become so clearly visible. As in a kaleidoscope, this banking crisis reveals the corruption and nepotism in all colours of the political spectrum.

\subsection*{Formation}

The BGB was formed in 1994 by unifying several Berlin credit institutes formerly controlled by the Berlin state government. At that time, the CDU/SPD coalition in the Senate boasted that the BGB, floated on the stock market, was unique in German banking, and not only served to strengthen Berlin as a banking centre, but also was aimed at carrying through a “trend-setting structural policy” in the city and its environs.
\subsection*{Ownership}
The largest single BGB shareholder was the Berlin state government (56.6 percent), 20 percent was held by the north German state bank NordLB, 7.5 percent belonged to the Gothaer insurance company (Parion) and 18.4 percent of the shares were in smaller holdings.

\subsection*{Company Profile}

With a balance sheet of almost 405 billion marks in the first quarter of the current year, the BGB was the largest financial company in Germany’s capital city. The BGB owned a series of finance houses, including the Landesbank Berlin, the BerlinHyp building society as well as a number of smaller banks like the Weber Bank and the Allbank. The various BGB enterprises employed 16,000 workers, but 3,000 were be dismissed by the end of 2001 due to the crisis. The BGB operated about 2.5 million private customer accounts and about 800,000 business accounts.

\subsection*{People}

The boss of the BerlinHyp was none other than one Klaus Landowsky, who was also the leader of the CDU parliamentary group in the Senate. The banker Landowsky agreed a 600 million mark line of credit, without the usual collateral and checks, to the two managing directors of the Aubis real estate company, whom the politician Landowsky knew all too well. They were not just businessmen but old CDU colleagues, who had sat alongside him for many years in parliament
.
From 1984 to 1990, Klaus-Hermann Winhold belonged to the CDU’s Berlin state leadership and his Aubis business partner Christian Neuling represented the Berlin CDU in the Bundestag (federal parliament). Neuling is not completely unknown to the public prosecutor’s office. He was investigated in 1985 on suspicion of selling waste oil as fuel oil. But Berlin public prosecutor’s office soon dropped all its proceedings against the prominent CDU politician.

In 1991, Neuling sat on the supervisory board of the Treuhand—established to oversee the sale of all East Germany’s former state-owned assets—and was chairman of the Bundestag’s Treuhand subcommittee when he was suspected of committing fraud. His company received properties and credits from the Treuhand on extremely favourable terms. Once again, Neuling denied everything, but resigned less than two weeks later as chairman of the parliamentary Treuhand subcommittee; and the investigation was shelved.

\subsection*{Aubis}

At the beginning of 2001, the Federal Banking Supervisory Office conducted a special audit of the BGB. As a result, it came to light that in 1995 Klaus Landowsky had personally received a CDU party donation of 40,000 marks in his executive office at the BerlinHyp and had not recorded the transaction correctly. The donors were Winhold and Neuling from the Aubis real estate company, who “shortly thereafter” were granted the 600 million mark credit by the BerlinHyp. According to the auditors’ comments published so far, “unprofessional and possibly illegal circumstances” prevailed.

When the Aubis bankruptcy loomed in 1999, the bank negotiated a financial disencumbrance in order to prevent the bankruptcy of the real estate company. The bank took over the rights to use the predominantly empty properties, in return forsaking repayment of the credit. In the context of this arrangement, the repayment of private loans to the Aubis managing directors and CDU politicians Wienhold and Neuling—worth more than five million marks—was also expressly dispensed with.

Landowsky stated again and again that he did not have to do anything with such million mark gifts and had not been involved in Aubis’s renovation efforts. However, Berlin’s Taz newspaper quoted a letter to Landowsky dated January 20 2000, which made clear that the CDU leader was well informed about the whole Aubis proceedings: “The private loans to Messrs N. and W. were covered by the purchase of the usufructuary right [the right to use and profit from another’s property]”.

In February 2001, Landowsky was forced to give up his position on the executive board of BerlinHyp, but continued to draw his lucrative income. He received two full years’ salary worth 1.4 million marks and will enjoy a monthly pension of almost 30,000 marks for the rest of his life for his work at BerlinHyp. In the middle of May, he was also forced to resign as a chairman of the CDU parliamentary group, but ensured he was elected deputy regional chairmen at the following CDU state convention. However, in view of the increasing pressure, he has already had to vacate this post.


%BgB
%Management of BgB comprised prominent politicians of the Christian Democratic Party.
\end{document}