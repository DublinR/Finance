Chinese Aviations (CAO Singapore)

2005/2006
\begin{itemize}
\item Created in 1993 , based in Singapore
\item Subsidary of CAO, which imported Kerosene
\item Biggest Trading Scandal since Nick Leeson 10 years previously
\item Losses of USD 550 Million
\end{itemize}

Key Person : Chen Jiulin

enters contract on assumption that oil will  not surpass USD 88 per barrell.

In October 2004, Oil surpassed $\$88$ per barrel leaving Chen and CAO exposed

March 2006
Chen Jiulin was convicted of insider trading and was sentenced to imprisonment.

Causes 
\begin{itemize}
\item no risk management policies in place
\item no oversight from senior management
\item erroneous financial statements
\end{itemize}

%============================================================================================================%
\section{China Aviation Oil }
\noindent China Aviation Oil (Singapore) Corporation Ltd (CAO) is
the Singapore subsidiary of China Aviation Oil. It was first
incorporated in 1993 and mainly deals in jet fuel (kerosene)
procurement for the airports in the People’s Republic of China and
international oil trading. The firm commands a near $100\%$ market
share of the procurement of imported jet fuel for China's civil
aviation industry.
\\
\\
The company caught public attention in 2005 when it was embroiled
with a trading scandal, involving its chief executive Chen Jiulin
with losses running up to $\$550$ million and the subsequent
collapse of the company. This made it one of the biggest business
scandals in Asia since the Nick Leeson case with the $\$1.2$
billion loss that sent Barings Bank into bankruptcy in 1995.
\\
\\
Subsequently, Chen Jiulin was arrested with the charge of insider
trading, and was sentenced to 51 months imprisonment. China
National Aviation Fuel Group Corporation (CNAF) its parent
company) has since came up with plans to revive the company.
\\
\\
Initially, only swaps and futures were traded to help optimise
CAO's procurement capabilities. Later, on behalf of client airline
companies, back-to-back option trading was started. Speculative
trading in fuel options started a year after that.
\\
\\
At first, the CAO primarily used derivatives as a hedging
instrument to hedge its risk inherent in its primary business of
physical oil procurement and trading. However, as stated in the
PwC report, at the end of March 2003, CAO entered into speculative
option trades where it sought to profit from favourable market
movements.\\\\ Strategically, CAO took the view that the market
price for oil would continue its trend upwards. For its trades in
options, it purchased calls and sold puts, thereby effectively
creating a geared long position. As oil prices increased, the
calls that were purchased were exercised and profits were made.
The puts that were sold were not exercised and the company
profited from the premiums that had been collected when these
options were sold. This strategy was extremely successful until Q3
2003. \\\\ In Q4 2003, CAO adopted a bearish view of the trend in
oil prices and as a result changed its trading strategy, according
to the PwC report. Chen Juilin signed contracts with multiple
banks and made a speculative bid on oil for $\$38$ per barrel with
the assumption that the oil price would not rise above that price.
It now sold calls and bought puts with the result that it was in a
short position at the end of 2003. \\\\ However, the company’s
strategy started to unravel when oil prices did not decrease from
the end of 2003. In October 2004, the situation came to a head
when international oil prices grossly surpassed the $\$38$ price,
leaving CAO facing significant margin calls on its open (short)
derivative positions.
\end{document}


China Aviation Oil

China Aviation Oil (Singapore) Corporation Ltd (CAO) is the Singapore subsidiary of China Aviation

Oil. It was first incorporated in 1993 and mainly deals in jet fuel (kerosene) procurement for the

airports in the People’s Republic of China and international oil trading. The firm commands a near

100% market share of the procurement of imported jet fuel for China's civil aviation industry.

The company caught public attention in 2005 when it was embroiled with a trading scandal,

involving its chief executive Chen Jiulin with losses running up to $550m and the subsequent

collapse of the company. This made it one of the biggest business scandals in Asia since the Nick

Leeson case with the $1.2 billion loss that sent Barings Bank into bankruptcy in 1995.

Subsequently, Chen Jiulin was arrested with the charge of insider trading, and was sentenced to 51

months imprisonment. China National Aviation Fuel Group Corporation (“CNAF”  ‐  its parent

company) has since came up with plans to revive the company.

Initially, only swaps and futures were traded to help optimise CAO’s procurement capabilities. Later,

on behalf of client airline companies, back‐to‐back option trading was started. Speculative trading in

fuel options started a year after that.

At first, the CAO primarily used derivatives as a hedging instrument to hedge its risk inherent in its

primary business of physical oil procurement and trading. However, as stated in the PwC report, at

the end of March 2003, CAO entered into speculative option trades where it sought to profit from

favourable market movements. Strategically, CAO took the view that the market price for oil would

continue its trend upwards. For its trades in options, it purchased calls and sold puts, thereby

effectively creating a geared long position. As oil prices increased, the calls that were purchased

were exercised and profits were made. The puts that were sold were not exercised and the company

profited from the premiums that had been collected when these options were sold. This strategy

was extremely successful until Q3 2003.

In Q4 2003, CAO adopted a bearish view of the trend in oil prices and as a result changed its trading

strategy, according to the PwC report. Chen Juilin signed contracts with multiple banks and made a

speculative bid on oil for $38 per barrel with the assumption that the oil price would not rise above

that price. It now sold calls and bought puts with the result that it was in a short position at the end

However, the company’s strategy started to unravel when oil prices did not decrease from the end

of 2003. In October 2004, the situation came to a head when international oil prices grossly

surpassed the $38 price, leaving CAO facing significant margin calls on its open (short) derivative

According to a CAO press release dated the 30 November 2004, “it was unable to meet some of the

margin calls arising from its speculative derivative trades, resulting in the company’s being forced to

close the positions with some of its counter parties”. In the same press release, CAO announced that

the accumulated losses from these closed positions amounted to approximately US$390 million. In

addition, the company had unrealized losses of about US$160 million, bringing the total derivative

losses to $550 million. However Chen Juilin and other senior executives manipulated the company’s

financial statements to conceal the losses.

Copyright ©The Professional Risk Managers’ International Association 

1

    China Aviation Oil

China Aviation Oil reports under accounting standards prevailing in Singapore, which is one of 100

countries worldwide that have effectively adopted IFRS. Prior to 2003, companies reported under

Singapore Statements of Accounting Standards (SAS). As part of SAS, the country's regulator

proposed SAS 33: Financial Instruments: Recognition and Measurement. This was due to be effective

for financial statements covering periods on or after 1 July 2001. However, this was subsequently

delayed to 2004. Then, as of 1 January 2003, all Singapore companies moved to Singapore Financial

Reporting Standards (FRS), which are effectively equivalent to IFRS.  

As in Europe, Singapore companies were required to adopt FRS 39 (the IAS 39 equivalent) only as of

1 January 2005. However, as of its initial public offering in 2001, China Aviation Oil, in the notes to its

accounts, stated that “Financial instruments undertaken for trading purposes are marked to market

and the gain or loss arising is recognised in the profit and loss account”.

PwC states that “for the accounting periods under review, the FRS did not specifically prescribe a

method of recognising and measuring derivatives. However, given its adoption of FRS and in the

absence of specific guidance on derivatives, the company ought to have adopted an accounting

treatment and valuation method for derivatives that accorded with industry standards” in its report

PwC concludes that CAO adopted an incorrect valuation methodology for its options. It regarded the

intrinsic value (i.e. difference between the strike price and the forward price of the underlying

commodity) as the fair value of its options. However, the fair value of options should comprise both

intrinsic value and time value, which takes into account the length of the time to maturity of the

option, the volatility in the spot price of the underlying commodity, interest rates and other factors.  

The majority of option valuation models include time value in their formulae; which most

commentators consider to be more appropriate valuations than the intrinsic value method adopted

PwC observes that if the time value had been taken into account, the company’s financial

statements would have reported a more realistic picture of the situation and it shows this by

recalculating the results of CAO’s option portfolio on what PwC considered a more representative

fair value basis. They go on to conclude that the “company’s reported earnings in 2004 were

TABLE1: China Aviation Oil – reported versus adjusted earnings (2004, $m)

Q1 Q2 Q3 First 9 months

Reported PBT 19.0 19.3 11.3 49.6

Adjusted PBT ‐6.4  ‐58.0  ‐314.6  ‐379.0

Source: PriceWaterhouseCoopers Report, 3 June 2005

Due to the continuous rise of the oil price during 2004, even a fair value approach based purely on

intrinsic value should have shown significant losses at China Aviation Oil in 2004. However, according

to PwC’s report, to avoid recording and reporting losses, the company adopted a much larger risk

exposure by selling long‐term options with extremely high risk profiles to raise the premiums to

cover the cost of closing out the loss‐making option contracts. So in effect, CAO covered‐up the

losses that were realised when closing out the loss‐making near‐dated options.

Copyright ©The Professional Risk Managers’ International Association 

2

    China Aviation Oil

• March 3002 – CAO enters into speculative option trades on oil prices with a bullish view

• Q4 2003 – CAO changed its strategy and started trading speculative option trades taking a

• October 2004  ‐ international oil prices rose steeply, leaving CAO facing significant margin

calls on its open (short) derivative positions.  

• 30 November 2004 – in a press release CAO states it was unable to meet some of the margin

calls arising from its speculative derivative trades, resulting in the company’s being forced to

close the positions with some of its counter parties. The accumulated losses from these

closed positions amounted to approximately US$390 million. In addition, the company had

unrealized losses of about US$160 million, bringing the total derivative losses to $550

• March 2006 ‐ Chen Jiulin was arrested with the charge of insider trading, and was sentenced

to 51 months imprisonment.

There were no properly defined risk management policies in place when the speculative option

trading started and there was a lack of oversight by the senior management.

As discussed above the option contracts, some of which had complex features such as optional term

extensions, were not valued according to best practice. As PwC states the time value was not taken

into account. CAO continued with its valuation approach (using only intrinsic value), even though the

confirmations of counterparties contained significantly different prices. This led to erroneous

In addition to the above, there were several roll‐overs of loss generating positions, whereby options

on larger volumes were being sold to generate sufficient funds to settle the existing position losses.

Was option valuation to blame?

Some may argue that CAO was acting correctly by valuing derivatives at intrinsic value, although that

still fell short of the full fair value approach as now required under IFRS and US GAAP. The China

Aviation Oil situation view, provides support for the new approach to derivatives accounting both

Historically, derivatives were generally not shown on balance sheets and gains and losses were only

accounted for in earnings upon settlement of the derivatives. Consequently investors do therefore

not have sufficient insight into the current position or risk exposure of the company.

It seems that the fair value measurement is but a snapshot at a random moment in time. Therefore

the changes introduced by the “IFRS 7 Financial Instruments – Disclosures” that requires disclosure

of the significance of financial instruments for a company’s financial position and performance are

CAO should have analysed the implications of different oil prices on the company’s results and

equity. This sensitivity analysis as required under IFRS 7 should provide investors and analysts with

Copyright ©The Professional Risk Managers’ International Association 

3

    China Aviation Oil

insight into the dynamics of value changes and the sensitivity of fair value to underlying drivers

(interest rates, exchange rates, equity prices, commodity prices etc.).

There were many issues and problems exposed with the CAO scandal, for example, corporate

governance, market monitoring and surveillance, compliance and regulation, inadequate knowledge

in risk management, wrong perception of the risks, and miscalculation of the depth of pocket.

There are many lessons to be learned from this case:

• The strategic objectives and risk management policy of the company must be determined at

the highest level in the organisation

• Senior management must be responsible for the policies to be incorporated into daily

operations and to dedicate the necessary resources to achieve this

• A risk management department must be set up to:

o ensure the open position is followed up;

o ensure compliance of the positions and strategies with the policies and procedures;

o determine other risks related to the activity, e.g. credit risk.

• Internal and external reporting (financial statements) must follow best practice with

frequent and detailed disclosures.

• There should be regular stress testing on strategy and risk measurement models

• Prior to trading in new products, a formal analysis must be conducted and documented on

the risk profile of the product, the appropriateness of its use, the ability to handle the new

product operationally, including the accounting treatment.

Derivatives disasters, such as the one of China Aviation Oil, have nothing to do with the inherent

risky character of derivative instruments. In CAO’s case the losses from the speculative oil derivative

trading were due in part to the company’s desire not to disclose losses in 2004, as well as improper

risk management procedures and the failure of the board to fulfil their duties.

Report by PwC on its investigation into the oil trading losses ‐ available from the China Aviation Oil website  (go

to news release for June 3rd, 2005).

Deloitte, “The China Aviation Oil Debacle”, published January 2006.

Serge Matulich, David M. Currie, “Handbook of Frauds, Scams, and Swindles”, published in 2008.

Alan N. Peachey, “Great Financial Disasters of our Time”, published in 2006.
