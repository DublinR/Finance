Daiwa Bank (1995)
====================
- Principal Person: Toshuhide Iguchi

- Iguichi lost $1.1 Billion in dealings on US treasury bonds over a period of 11 years.

- Iguchi used his position of NY Branch's custody department to cover losses. He sold off assets owned by Daiwa and Customers.

- Daiwa was big enough to survive these losses, but was kicked out of the market.

- Daiwa bank lost huge amounts of credibility in attempting to cover up its NT activities.

- Daiwa and its auditors never independently confirmed the custody account statements.


Daiwa Employee "Toshihide Iguchi" confessed that he had lost around $\$1.1$ billion while dealing with US treasury bonds, in a letter to Daiwa Bank President

 - Iguchi was executive vice president of Daiwas NY branch
 - Losses occurred over 11 years
 - used his position as head of branches security  custody department to cover the losses by selling securities owned by Daiwa and its customers.

Coverup by Iguchi and then by banks officiers was reported to the federal reserve board. This resulted in one of japan's biggest commercial banks being kicked out of the 
the US market.

Daiwa $200 billion is assets and $8 billion in reserves meant that it was large enough to take the hit. But punishment and humiliation dealt massive blow to comanies 
reputation.


%====================================================%
\section*{Criminal Proceedings}

Criminal Charges were brought against the bank.


In February 1996 Daiwa agreed to pay $340 billion fine (a record criminal fine) as way of settling the case.

Iguchi was sentenced to 4 years and $2.6 million penalty, which he has little hope of paying.
