Riggs opened an account for the equatorial guinea government to recieve funds frol Oil Companies doing business
in Eq. guinea under terms allowing withdrawals with two signatures. One from the EG president
the other from 	- his son, the E.G. Minister of Mines 
 		- his Nephew, the EG secretary of state for Treasury and budget

Riggs bank assisted Gen. Pinochet , former president of Chile, to evade legal proceedings related to his Riggs Bank
account and resisted OCC oversight of these accounts. This was despite red flags ivloving the source of Mr. Pinochets 
Wealth, penind legal proceedings to freeze his assets, and Public allegations of serious wrongdoing by him.

Riggs Bank actually conducted transactions through Rigg's own accountto hide Mr. Pinochet's involvement in some cash transactions.

By 2003, the EG account represented the largest relationship at Riggs Bank, with aggregate deposits from $400 and $700 million at the time.


 
Riggs Bank


 
In July 2004, the much respected Riggs Bank, (Riggs) the oldest bank in Washington, established in
1840 by George Washington Riggs and banker to Abraham Lincoln himself, sprang to unwanted
prominence when it was revealed that it had been fined $25 million in May 2004 for not reporting
suspicious transactions in accounts by the dictator of Equatorial Guinea, the former president of
Chile Augusto Pinochet, and by the Saudi Arabian ambassador to the United States.
 
A report released by Senate investigators in July 2004 stated that Riggs maintained accounts for Augusto
Pinochet, the former Chilean dictator, long after a Spanish court in 1998 had requested that such
accounts be frozen. The report stated that Riggs had maintained accounts worth $4 - 8
million for Mr Pinochet between 1994 and 2002 and had helped him set up offshore shell companies to disguise his control of the accounts.
 
In the case of Equatorial Guinea (EG) the bank had handled deposits and loans of nearly $700
million; indeed EG had been the bank’s largest customer until the more

- than 60 accounts, some of which had been first opened in 1995, had been closed over the previous several months. The investigation had discovered that between 2000 and 2002 more than $11.5 million – in cash – had
arrived in suitcases and had been deposited in accounts controlled through offshore corporations established in the Bahamas with Riggs’ assistance; which the bank had not reported accurately to the US regulators. In other cases, wire transfers totalling almost $35 million had been sent from EG’s
oil accounts at the bank to unknown companies in banking secrecy havens.
 
PNC
Subsequently, on 16

th July, 2004 it was announced that the Pittsburgh-based PNC Financial had
agreed a $800 million purchase of Riggs; however although Riggs had already closed many of its
embassy accounts, PNC was insisting that all embassy business must be divested before the takeover
could be implemented. The effect of this was that banks were shying away from low

-profit embassy
banking and other business perceived to be financially risky (and with reputational risks). It was also
reported that in June of that year, US regulators had met with Citigroup, Bank of America, and other
banks to try and help the embassies of Saudi Arabia and other countries find alternative banking
facilities.
In September 2004 the bank started closing banks accounts of London
-based embassies
held in its branch. The bank was estimated to maintain accounts for two

-thirds of the embassies in
London and for thousands of their staff.
 
The Senate Investigation - poor AML compliance
The evidence reviewed by the Senate Subcommittee staff established that, since at least 1997, Riggs
has disregarded its’ Know

-Your-Customer (KYC) and anti-money laundering (AML) obligations,
maintained a dysfunctional AML program despite frequent warnings from Office of the Controller of
Currency (OCC) regulators, and allowed or, at times, actively facilitated suspicious financial activity.
The evidence also showed that federal regulators did a poor job of compelling Riggs to comply with
statutory and regulatory anti-

money laundering requirements. They were tolerant of the bank’s
weak AML program, too slow in reacting to repeat deficiencies, and failed to make prompt use of
available enforcement tools.
Two sets of Riggs Bank (Riggs) accounts, one involving Augusto Pinochet and the other involving
Equatorial Guinea, illustrated the bank’s poor AML compliance. They also illustrated the failure of
federal bank regulators to exercise meaningful oversight of a bank with numerous high risk accounts
and fundamental, long

-standing AML deficiencies.


Assisting Augusto Pinochet.

The evidence obtained by the Subcommittee staff showed that, from 1994 until 2002, Riggs opened
at least six accounts for Augusto Pinochet (former President of Chile) and issued several certificates
of deposit (CDs) while he was under house arrest in the United Kingdom and his assets were the
subject of court proceedings. The aggregate deposits in the Pinochet accounts at Riggs ranged from
$4

-$8 million at a time. The Subcommittee investigation had determined that the bank’s leadership
directly solicited the accounts from Mr. Pinochet, and Riggs account managers took actions
consistent with helping Mr. Pinochet to evade legal proceedings seeking to discover and attach his
bank accounts. The investigation found that Riggs opened multiple accounts and accepted millions
of dollars in deposits from Mr. Pinochet with no serious inquiry into questions regarding the source
of his wealth; helped him set up offshore shell corporations and to open accounts in the names of
those corporations to disguise his control of the accounts; altered the names of his personal
accounts to disguise their ownership; transferred $1.6 million from London to the United States
while Mr. Pinochet was in detention and the subject of a court order to attach his bank accounts;
conducted transactions through Riggs’ own accounts to hide Mr. Pinochet’s involvement in some
cash transactions; and delivered over $1.9 million in cashier’s checks to Mr. Pinochet in Chile to
enable him to obtain substantial cash payments from banks in that country.
 
The OCC
The Subcommittee investigation also determined that Riggs concealed the existence of the Pinochet
accounts from OCC bank examiners for two years, initially resisted OCC requests for information,
and closed the accounts only after a targeted OCC examination in 2002. Despite Riggs’ track record
of repeat AML deficiencies, the OCC’s concern about the Pinochet accounts, and Riggs’ concealment
of them from the agency, the OCC took no enforcement action against the bank after it learned of
those actions in 2002.
 
Moreover, in July 2002, the OCC Examiner
-in-Charge at Riggs instructed the examiners who had
investigated the Pinochet accounts not to include their examination memorandum or supporting
work papers in the OCC’s electronic files for Riggs. The Subcommittee learned that such an
instruction was highly unusual and contrary to OCC procedure and practice. And about a month
later, the OCC Examiner

-in-Charge accepted a job at Riggs.
 
Equatorial Guinea Accounts.
The Subcommittee investigation also determined that, from 1995 until 2004, Riggs administered
more than 60 accounts and CDs for the government of Equatorial Guinea (E.G.), E.G. government
officials, or their family members.
 
By 2003, the E.G. accounts represented the largest relationship at Riggs, with aggregate deposits
ranging from $400 to $700 million at a time. The Subcommittee investigation had determined that
Riggs serviced the E.G. accounts with little or no attention to the bank’s anti

-money laundering obligations, turned a blind eye to evidence suggesting the bank was handling the proceeds of foreign
corruption, and allowed numerous suspicious transactions to take place without notifying law
enforcement. The Subcommittee investigation found, for example, that Riggs opened multiple
personal accounts for the President of Equatorial Guinea, his wife, and other relatives; helped
establish shell offshore corporations for the E.G. President and his sons; and over a three

-year period, from 2000 to 2002, facilitated nearly $13 million in cash deposits into Riggs accounts
controlled by the E.G. President and his wife. On two of those occasions, Riggs accepted without due
diligence $3 million in cash deposits for an account opened in the name of the E.G. President’s
offshore shell corporation.


 
In addition, Riggs opened an account for the E.G. government to receive funds from oil companies
doing business in Equatorial Guinea, under terms allowing withdrawals with two signatures, one
from the E.G. President and the other from either his son, the E.G. Minister of Mines, or his nephew,
the E.G. Secretary of State for Treasury and Budget. Riggs subsequently allowed wire transfers
withdrawing more than $35 million from the E.G. government account, wiring the funds to two
companies which were unknown to the bank and had accounts in jurisdictions with bank secrecy
laws. The Subcommittee has reason to believe that at least one of these recipient companies is
controlled in whole or in part by the E.G. President. When, in 2004, the bank requested more
information about the two companies from the E.G. President, he declined to provide it, except to
say the wire transfers to them had been authorized.
The senior leadership at Riggs were well aware of the E.G. accounts and had met on several
occasions with the E.G. President and other E.G. officials. The bank leadership permitted the account
manager handling the E.G. relationship to become closely involved with E.G. officials and business
activities, including advising the E.G. government on financial matters and becoming the sole
signatory on an E.G. account holding substantial funds. The bank exercised such lax oversight of the
account manager’s activities that, among other misconduct, the account manager was able to wire
transfer more than $1 million from the E.G. oil account at Riggs to another bank for an account
opened in the name of an offshore corporation controlled by the account manager’s wife.
In response to a Subcommittee subpoena, Riggs initially failed to identify a number of E.G. accounts
at the bank. The Subcommittee later learned that the bank had failed to designate any of the E.G.
accounts as high risk accounts until October 2003, and did not subject them to additional scrutiny
despite obvious warning signs, such as the involvement of foreign political figures, a country with a
culture of corruption, and frequent high dollar transactions.
The bank also failed to monitor or report suspicious activity in the E.G. accounts.
The bank closed all these accounts in 2004.
Riggs’ Dysfunctional AML Program.
The evidence demonstrated that the Pinochet and E.G. accounts were not treated in an unusual
manner, but were the product of a dysfunctional AML program with long

‐standing, major
deficiencies. These deficiencies included the inability to readily identify all of the accounts associated
with a particular client, the absence of any risk assessment system to identify high risk accounts,
inadequate client information, the lack of an established policy for handling accounts associated
with foreign political figures, the failure to provide enhanced monitoring of high risk accounts, the
failure to monitor wire transfer activity, the failure to detect and report suspicious activity, untimely
and incomplete internal audits, and inadequate AML training. These flaws were repeatedly identified
in regulatory examinations and internal audits, and Riggs repeatedly promised to correct them, but
failed to do so.
Regulatory Failure.
Given the fundamental, long standing deficiencies in Riggs’ AML program, it is difficult to understand
why federal regulators failed to act sooner to require the bank to correct them. The OCC recently
acknowledged: “there was a failure of supervision” at Riggs, and “we gave the bank too much time.”
The evidence shows that, since 1997, OCC examiners repeatedly identified major AML deficiencies at
Riggs, but more senior OCC personnel allowed these AML deficiencies to continue year after year
without forceful action to stop them.
 
In the case of Riggs, the evidence also indicated that the OCC’s Examiner

-in-Charge (EIC) appeared to have become more of an advocate for the bank than an arms-length regulator. In 2001, for example,
he advised more senior OCC personnel against taking a formal enforcement action against Riggs,
because the bank had promised to correct identified AML deficiencies. In 2002, he ordered
examiners not to include a memorandum or work papers on the Pinochet examination in the OCC’s
electronic database. About a month after giving this order, that same examiner was hired by Riggs,
creating an appearance of a conflict of interest. During his tenure at the bank, he attended a number
of meetings with OCC personnel related to Riggs’ AML problems. Federal law bars former federal
employees from appearing before their former agencies on certain matters, and OCC rules bar
former OCC employees from even attending meetings with the agency for two years, unless the OCC
ethics office approves the contact.
Despite these post

‐employment restrictions, the former Riggs examiner failed to obtain clearance
from the OCC ethics office prior to attending the meetings with OCC personnel. These actions –
•

advising against a formal enforcement action,
•

suppressing the Pinochet examination materials,
•

accepting a job offer at the bank he regulated, and
•

ignoring post‐employment restrictions on OCC contact;
all of which suggest that the Examiner had become much too close to Riggs during the years he was
responsible for overseeing it.
In addition, the facts demonstrated that his supervisors were too slow in reacting to repeat
deficiencies at the bank and were too reluctant to make use of available enforcement tools to
compel AML compliance.
•

In 2001, for example, when presented with three examination reports outlining AML
deficiencies at Riggs, OCC enforcement personnel went along with the EIC’s
recommendation against taking any enforcement action.
•

In 2002, after learning that Riggs had hid the Pinochet accounts from the agency for two
years and facilitated suspicious transactions, OCC supervisors, again, failed to take any
enforcement action. The OCC failed even to issue a final examination report on the Pinochet
matter.
•

In 2003, after uncovering extremely troubling information in connection with accounts
associated with Saudi Arabia, the OCC took its first enforcement action against the bank,
issuing a Cease and Desist order requiring it to revamp its AML program. This order was
more comprehensive and capable of enforcement in court than directives in prior
examination reports, but included no punitive measures at the time such as a civil fine.
•

It was only in 2004, six years after the OCC began citing Riggs for AML deficiencies, that
federal regulators imposed their first civil fine on the bank.
The key OCC enforcement actions against Riggs also took place after negative press reports began
raising public questions about Riggs’ AML safeguards. For example, the OCC’s in-depth review of the
Saudi accounts followed press articles that began appearing in November 2002, suggesting links
between certain Riggs accounts and the 9-11 terrorist attack. This examination resulted in the OCC’s
identifying the same deficiencies as in earlier years, but in contrast to the agency’s prior willingness
to rely on promises by the bank to improve, the OCC issued a public Cease and Desist order requiring
corrective action. The OCC’s examination of the E.G. accounts in 2003 and 2004 was, in turn,
prompted by a negative press article in January 2003 suggesting these Riggs accounts were being
misused by E.G. officials and by the Subcommittee’s investigation of these accounts throughout
2003. The OCC has indicated that it was the E.G. examination that opened their eyes to still more
bank misconduct and to evidence of the bank’s utter failure to implement promised AML reforms,
resulting in the decision to impose a civil fine on the bank.
The Subcommittee’s investigation indicated that the failure of supervision in the Riggs matter was
not an isolated case, but symptomatic of a pattern of uneven and, at times, ineffective AML
enforcement by federal regulators. The General Accounting Office had summarized a number of
cases in addition to Riggs showing that federal regulators have allowed AML compliance problems to
persist for years without correction. These cases indicate that all of the federal financial regulators,
not just the OCC, need to strengthen their AML enforcement efforts by requiring prompt correction
of identified AML deficiencies, making greater use of formal enforcement tools when financial
institutions ignore their AML obligations, and issuing more timely civil fines. Regulators should also
consider developing a policy requiring mandatory enforcement action within a specified period of
time against any financial institution with major, repeat AML violations.
An important ancillary issue raised by the Riggs case history involved the ability of U.S. financial
institutions with foreign affiliates to get key due diligence information about accounts opened and
managed by their foreign affiliates.
Oil Company Payments.
During its analysis of large bank transactions involving E.G. accounts at Riggs and other financial
institutions, the Subcommittee staff became aware of a number of substantial payments that had
been made by oil companies doing business in Equatorial Guinea to individual E.G. officials, their
family members, or entities controlled by these officials or family members. These payments, which
sometimes exceeded $1 million, paid for E.G. land leases or purchases, E.G. Embassy expenses, incountry
security services, or expenses for E.G. students studying abroad. In a few instances, the
evidence shows that oil companies entered into business ventures with companies owned in whole
or in part by the E.G. President, other E.G. officials, or relatives. For example, in 1998, ExxonMobil
established an oil distribution business in Equatorial Guinea of which 85 percent is owned by
ExxonMobil and 15 percent by Abayak S.A., a company controlled by the E.G. President.
Governance Risk Lessons to be Learnt
1. It is easy and tempting for banks with client or geographical limitations to move into riskier
business lines without giving (or completely ignoring) the risks involved.
2. Even the most relaxed regulatory authorities will not tolerate continued flaunting of
regulations designed to prevent crime, money laundering, and terrorism – especially when
politically sensitive.
3. The “risk management” tone is set by the top management, which in this case was
overwhelmingly below standard.
4. Complete and accurate record keeping, tight risk management regimes, and an independent
compliance function are the benchmarks required by all banks.
5. The timely application of regulatory sanctions would have prevented the continuation of the
laxity in exercising adequate governance over these high profile, and high risk, accounts.
Extracted from the report prepared by the Minority Staff of the United States Senate Permanent
Subcommittee on Investigations on Government Affairs relating to Money Laundering and Foreign
Corruption: Enforcement and Effectiveness of the Patriot Act Case Study Involving Riggs Bank, and
released in conjunction with the Permanent Subcommittee on Investigations’ hearing on July 15

th
2004; and from “Great Financial Disasters of our Time” by Alan N. Peachey, ISBN 3

‐8305‐1162‐0.
