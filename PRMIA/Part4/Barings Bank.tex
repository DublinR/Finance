\documentclass[PRMIA4A.tex]{subfiles} 

\begin{document} 
\newpage
\section{BARINGS BANK}

\subsection{Learning Outcomes}
\begin{itemize}
	\item Describe how the massive losses were incurred
	\item Describe why the true position was not noticed earlier
	\item Describe the role of the External Auditors
	\item Describe the supervision done by the Bank of England
	\item Describe the role of The Securities And Futures Authority
	(SFA, now knownas the Financial Services Authority, the FSA)
	\item Describe the Lessons learnt from the Barings Case Study
	\item Discuss the events leading up to the losses, the risks incurred and
	the mitigation processes described
\end{itemize}
	
%\section{Baring's Bank}
%----------------------------------------------%
Report published 1995

Asks key questions - how losses occured and why true position was not spotted before.

\subsection{Conclusions}

\begin{itemize}
\item Losses occurred due to unauthorised and concealted trading activities
\item Serious failures of control and managerial confusion at Barings aggravated the problem
\item Not detected prior by external auditors, supervisors or regulators
\end{itemize}
%----------------------------------------------%
\newpage
\subsection{Unauthorised Trading}

\begin{itemize}
\item Leeson had no authority to maintain open positions overnight, and had specific limits on intra-day trading
\item Leeson had no authority to trade in options (except as execution broker)

\item \textbf{Account 88888} was opened in July 1992 shortly after he was posted to Singapore

\item By 31 December 1994 he had accumulated losses of $\$$ 208 Million on this account. By 27 February 1995, it was as high as $\$$ 380 Million.

\item Leeson represented that he actually made profits, and was considered a star performer by Baring's Bank.
\item This profits were ``generated" by switching between SIMEX and Japanese Exchanges. The transactions were notionally risk free arbitrage transactions.
\item Unauthorised trading was funded by money advanced to BFS by BSI and BSLL.

\item Concealment was carried out by the following method
\begin{itemize}
\item Suppression of Account 88888 
\item Submission of falsified reports
\item Misrepresentation of profitablity
\item False Trading Transactions
\item False Accouting Entries
\end{itemize}
\end{itemize}
%------------------------------------------------%
\subsection{Peter Norris (COO Of BSL) }

In 1992 Norris started to introduce more controls in what was previously highly uncontrolled.

These controls didn't reach a satisfactory level, and were disparaged by the chairman, Peter Baring, as an absolute failure.

%------------------------------------------------%
\subsection{ Lack of Segregation of Lesson's Duties}

The fact that Lesson was in Charge of both the front and back office at BFS was a major failing.

Tony Hawes, the group treasurer, made his views known to James Baker, prior to Baker's Internal Audit of BFS in July/August 1994.

Ths audit recommended segregation of roles, but this was never implemented.

\end{document}






