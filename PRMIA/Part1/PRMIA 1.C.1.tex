
1.C.2 Global Markets and Terminology
1.C.3 Drivers of Liquidity
1.C.4 Liquidity and Financial Risk Management
1.C.5 Exchange v OTC Markets
1.C.6 Technology Change
1.C.7 Post Trade Processing
1.C.8 Retail \& Wholesale Brokerage
1.C.9 New Financial Markets
1.C.10. Conclusion
%===================================================================%
PRMIA 1C

A Swing Option in the electricity market gives the owner the right to use energy for a certain limit at a fixed price for a fixed times.

%===================================================================%


PRMIA

1.C.1.1 Introduction 
Global Markets and their terminology
Drivers of liquidity
Repo Markets
Technological Change
Post Trade Processing
Retail and Wholesale Brokerage
New Financial Markets
Conclusion

PRMIA 1.C.1 The Structure of Financial Markets 
Learning Outcome Statement
The candidate should be able to:
  Compare and contrast financial exchanges and OTC markets
  Define inter-dealer market and inter-dealer broker
  Compare and contrast the size of various markets (bonds, foreign exchange,equities, etc)
  Discuss the importance of market liquidity
  Describe a repo and a reverse repo and their roles as sources of liquidity
  Define an ISDA Master Agreement
  Describe how screen-trading systems work
  Describe a market “specialist”
  Describe an “open-outcry” trading system
  Describe an ECN
  Describe the steps in post-trade processing
  Describe straight-through processing
  Compare and contrast retail, wholesale and prime brokers
  Discuss issues with “new market” developments and structured products


PRMIA 1.C.1 The Structure of Financial Markets
Learning Outcome Statement
Section 1: Introduction
Section 2: Global Markets and Their Terminology
Section 3: Drivers of Liquidity
Section 4: Liquidity and Financial Risk Management
Section 5: Exchange v OTC Markets
Section 6: Technological Change
Section 7: Post trade Processing
Section 8: Retail and Wholesale brokerage
Section 9: New financial markets
Section 10: Conclusion

%--------------------------------------------------------------------------------------------------------%
Section 1: Introduction

A financial market is a mechanism that allows people to buy and sell (trade) financial securities (such as stocks and bonds), commodities (such as precious metals or agricultural goods), and other fungible items of value at low transaction costs and at prices that reflect the efficient-market hypothesis.

Section 2: Global Markets and Their Terminology

Section 3: Drivers of Liquidity

Liquidity is the degree to which an asset or security can be bought or sold in the market without affecting the asset's price. Liquidity is characterized by a high level of trading activity. Assets that can by easily bought or sold, are known as liquid assets.

3.1 Repo Markets
A repurchase agreement (or repo) is an agreement between two parties whereby one party sells the other a security at a specified price with a commitment to buy the security back at a later date for another specified price. Most repos are overnight transactions, with the sale taking place one day and being reversed the next day. Long-term repos—called term repos—can extend for a month or more. Usually, repos are for a fixed period of time, but open-ended deals are also possible. Reverse repo is a term used to describe the opposite side of a repo transaction. The party who sells and later repurchases a security is said to perform a repo. The other party—who purchases and later resells the security—is said to perform a reverse repo.

Section 4: Liquidity and Financial Risk Management

%-------------------------------------------------------------------------------------------------------%
Section 5: Exchange v OTC Markets
Some financial or commodities instruments are traded on established exchanges. Examples include most highly-capitalized stocks, which trade on exchanges such as the New York Stock Exchange, and futures, which trade on futures exchanges such as the Chicago Board of Trade. These instruments are called exchange traded.

An instrument is traded over-the-counter (OTC) if it trades in some context other than a formal exchange. Most debt instruments are traded OTC with investment banks making markets in specific issues. If someone wants to buy or sell a bond, they call the bank that makes a market in that bond and ask for quotes. Many derivative instruments, including forwards, swaps and most exotic derivatives are also traded OTC. In these markets, large financial institutions serve as derivatives dealers, customizing derivatives for the needs of clients.

Section 6: Technological Change
Section 7: Post trade Processing
Section 8: Retail and Wholesale brokerage

Section 9: New financial markets
Section 10: Conclusion



