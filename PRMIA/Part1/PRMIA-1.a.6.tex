
PRMIA 1.A.6 The Term Structure of Interest Rates


--------------------------------------------------------------------------------
Learning Outcome Statement

The candidate should be able to:
a.
  Describe yield to maturity as an internal rate of return

b.
  Define spot curve, spot rate and term structure

c.
  Define and describe the yield curve

d.
  Demonstrate the process of bootstrapping

e.
  Define no-arbitrage pricing

f.
  Calculate implied forward rates

g.
  Describe normal, flat and inverted yield curves

h.
  Describe the pure expectations theory

i.
  Describe the liquidity preference theory

j.
  Describe the preferred habitat theory

k.
  Describe the market segmentation theory

l.
  Compare and contrast the Ho-Lee, Hull-White and Black-Derman-Toy models

m.
  Compare and contrast single-factor and multi-factor models

n.
  Describe mean reversion

o.
  Calculate the value of non-callable bonds using term structure models

p.
  Describe the impact of an embedded call on the value of a bond using term structure models

q.
  Calculate effective duration and convexity within a term structure model

r.
  Define Option Adjusted Spread

s.
  Discuss the implications of choosing one term structure model over the others




--------------------------------------------------------------------------------


Term Structure Of Interest Rates Mean?

A yield curve displaying the relationship between spot rates of zero-coupon securities and their term to maturity.






The resulting curve allows an interest rate pattern to be determined, which can then be used to discount cash flows appropriately. Unfortunately, most bonds carry coupons, so the term structure must be determined using the prices of these securities. Term structures are continuously changing, and though the resulting yield curve is usually normal, it can also be flat or inverted.


1)

2)

3)

4)

5) 

6)

7) 

8) 




--------------------------------------------------------------------------------


1A61

Periodic interest rates versus effective annual yield.

Effective yield is the ratio of the absolute increase over the principal




Annual compounding versus more regular compounding

periodic interest rates vs Effective annual yield.


Term Structure: A Definition.




--------------------------------------------------------------------------------
1.A.6.2





--------------------------------------------------------------------------------
1.A.6.3 Shapes of Yield Curves


A yield Curve represents the relationship between yield and maturity at a specific point in time.


Inverted yield curves are unusual but not necessarily rare.


Shapes of the yield curve

a yield curve represents the relationship between yield and maturity at a specific point in time.

While this relationship changes over time there are three commonly yield curve shapes
•
normal

•
inverted

•
humped






--------------------------------------------------------------------------------
1.A.6.4

Spot and Forward Rates


spot rate : prevailing interest on a zero rate bond

short rate : one term interest rate that prevails at time t


Forward rates are marginal rates




--------------------------------------------------------------------------------
1.A.6.5


Term Structure Theories


1. Pure or Unbiased Expectations

2. Liquidity Preference (Liquidity Premium)

3. Market Segmentation


Expecations Theory (h)

One of the three main theories of term structure of interest rates.


The expectations theory argues that the investor should expect to receive the same return from any of the candidate strategy.


(Rising yield Curve: Interest rates are expected to rise)


The liquidity premium theory ( liquidity preference theory)(i)


History has shown that upward sloping yield curves tend to occur more frequently.


This theory states that investors demand a yield premium as compensation for investing long term.

This is because investors are risk averse and presumably believe that short term securities carry less risk that long term securities.

Thus a higher rate of return on long term securities is necessary to attract investors.


Preferred Habitat Theory (j)


The market segmentation theory explains term structures as a series of 'habitats' or preference zones which certain investors select depending on term to maturity.


For example, an investor with funds which are not needed for five years would select securities with a five year maturity.





--------------------------------------------------------------------------------
1.A.6.6


 - Compounding Rates

 - Importance of Yield Curves

 - Importance of Zero Rates


Interest rate calculations : compounding rates are fundamental.


Yield Curve: Important for Financial Engineers .Many assets pricing models are based on discounted cash flow analysis, whcih rely on cashflows and hence interest rates.


Bond prices and Derivate prices depend on a specific yield curve , the zero rate curve, for accurate pricing.


The zero rates are the only clean rates that reflect the time value of money. 

(Other rates being contaminated with coupon payments.)


Zero rates are needed to derive a fair price for an asset.


