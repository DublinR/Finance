
PRMIA 1.A.5 Basics of Capital Structure 



--------------------------------------------------------------------------------


Learning Outcome Statement


The candidate should be able to:
\begin{itemize}	
\item	  Characterize the impact of leverage on ROE volatility
\item	  Characterize the impact of taxes on the debt/equity decision
\item	  Compare the CFO considerations for issuing debt vs. equity
\item	  Describe the Agency costs of Debt
\item	  Describe the Agency costs of Equity
\item	  Describe the characteristics of Debt and Equity
\item	  Explain and Show the formula for the Value of a Firm
\end{itemize}	
% -------------------------------------------------------------------------------%
1) Introduction

2) Maximising shareholder value, incentives and Agency costs

3) Characteristics of debt and equity

4) Choice of Capital Structure

5) Making the Capital Structure Decision

6) Conclusion



% -------------------------------------------------------------------------------%
\subsection*{Agency Costs}

A type of internal cost that arises from, or must be paid to, an agent acting on behalf of a principal. Agency costs arise because of core problems such as conflicts of interest between shareholders and management. Shareholders wish for management to run the company in a way that increases shareholder value. But management may wish to grow the company in ways that maximize their personal power and wealth that may not be in the best interests of shareholders.





% -------------------------------------------------------------------------------%
\subsection*{Capital Structure}


A mix of a company's long-term debt, specific short-term debt, common equity and preferred equity. 


The capital structure is how a firm finances its overall operations and growth by using different sources of funds.


Debt comes in the form of bond issues or long-term notes payable, while equity is classified as common stock, preferred stock or retained earnings. 


Short-term debt such as working capital requirements is also considered to be part of the capital structure.


A company's proportion of short and long-term debt is considered when analyzing capital structure. When people refer to capital structure they are most likely referring to a firm's debt-to-equity ratio, which provides insight into how risky a company is. Usually a company more heavily financed by debt poses greater risk, as this firm is relatively highly levered.


Long Term debt are loans and financial obligations lasting over one year. In the U.K., long-term debts are known as "long-term loans."
% -------------------------------------------------------------------------------%



1A5
Basics of capital structure
introduction
maximising shareholder value

agency costs of debt
agency cost of equity
information assymmetries
characterostics
of debt and equity

Choice of capital structure
debt can be attractive
making the capital structure decision

greater flexibility

exposure to bankruptcy costs
exposure to financial distress costs

Guidelines

%==============================================================================%


\subsection*{1.A.5.1}
\begin{description}
\item[V:] value of the business, equal tot he value of the debt plus the value of the equity. (i.e. V = D + E)
\item[OFCF :]  Expected Operating Free Cash Flow after tax in period t
\item[WACC:] Weighted Average Cost of Capital. WACC is equal to the weighted average of the cost of debt and the cost of equity.
\end{description}


1) Discussion of agency Costs to shareholders that arise when a business appoints professional managers to operate on thier behalf.
 In particular, the incentives such managers have to transfer wealth from debt holders to shareholders.

2) Characterisation of Debt and Equity. Identification of some key variables that distinuish different forms of debt.

3) Choice of Capital Structure
Debt is not Cheap.
4) Debt Equity Mix


\subsection*{1.A.5.6 Conclusion}

Business is funded by combination of debt and equity.

Choice of capital structure affects value of business and so affects managers and shareholders.

A Key determinant of capital structure is volatility of operating cash flows.

The more volatile the risk to cashflows the less debt can be supported.





1A4
The CAPM and multifactor models
 systemic risk
 capital asset pricing model
 sharpe ratio

The CAPM provides an elegant model of the determi.ants of the equilibrium expected or required return on any individual risky asset.
 it then predicts that the expected return on a risky asset ER consists of a risk freerate r plus a risk premium

 The risk premium is proprtional to the excess market return with the constant of proprtionality being the beta of a risky asset.
The CAPM allows one to assess the relative volatility of the expected returns on the individual stocks on the basis of their beta values.

 a beta greater than one is said to be an aggressive stock. It moves more on average than the expected market return

conversely a beta of less than one is said to be defensive

investors rank beta to rank the relative safety.

Single Index Model

Multifactor Models and the APT
portfolio returns

The CAPM is a logical consequencevof the mean variance portfolio theory.
 and assumes the following
all investors have homogeneous expectations
investors coose their risky assets by maximising the sharpe ratio
investors can borrow or lend unlimited amounts at the risk free rate
the mwrket is in equilibrium at all times

It provides an equation that determines the return required by investors to willingly hold any particular risky asset as part of a well diversified portfolio.

 estimating beta
 can be obtained using ordinary least squarez time series regression.



\subsection*{Sharpe ratio}

The CAPM can be re arranged and expresses in terms of the SML. Suppose that the historic average value of the market risk premium and the risk free rate.

