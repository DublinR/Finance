
PRMIA 1.C.3 The Bond Markets

Learning Outcome Statement

The candidate should be able to:
a.
  Compare and contrast a retail and an investment bank

b.
  Define market-making and origination

c.
  Describe the various market participants by group

d.
  Compare and contrast a proprietary trader and a market-maker (dealer) and an inter-dealer broker

e.
  Define bid-price and offer-price

f.
  Compare and contrast sovereign, agency, corporate and municipal bonds

g.
  Describe on-the-run, off-the-run and benchmark securities

h.
  Compare and contrast general obligation and revenue bonds

i.
  Define a sinking fund

j.
  Define property clauses and call provision

k.
  Define types of foreign bonds (Yankee, Bulldog, Samurai, Alpine and Matador)

l.
  Compare gross and net interest payments

m.
  Compare and contrast the primary and secondary markets

n.
  Compare and contrast a public offer and a private offer

o.
  Describe the process of underwriting a new issue

p.
  Define underwriter, lead manager and book-runner

q.
  Define a fixed-price re-offer mechanism

r.
  Define a bought-deal

s.
  Describe the characteristics of the Eurobond market

t.
  Define the different day-count conventions

u.
  Define default and recovery rates

v.
  Describe how a bond’s rating affects the yield spread

w.
  Describe the role of Rating Agencies




--------------------------------------------------------------------------------


PRMIA 1.C.3 The Bond Markets

1. Introduction

2. The Players

3. Bonds by Issuers

1) Government Bonds

2) US Agency Bonds

3) Municipal Bonds

4) Eurobonds

5) Corporate Bonds


4. The Markets

5. Credit Risk

6. Summary




--------------------------------------------------------------------------------


Sinking Fund  (i)

A means of repaying funds that were borrowed through a bond issue. 

The issuer makes periodic payments to a trustee who retires part of the issue by purchasing the bonds in the open market.


Rather than the issuer repaying the entire principal of a bond issue on the maturity date,  another company buys back a portion of the issue annually and usually at a fixed par value or at the current market value of the bonds, whichever is less. Should interest rates decline following a bond issue, sinking-fund provisions allow a firm to lessen the interest rate risk of its bonds as it essentially replaces a portion of existing debt with lower-yielding bonds.


From the investor's point of view, a sinking fund adds safety to a corporate bond issue: with it, the issuing company is less likely to default on the repayment of the remaining principal upon maturity since the amount of the final repayment is substantially less. This added safety affects the interest rate at which the company is able to offer bonds in the marketplace.





--------------------------------------------------------------------------------


Foreign Bonds (k)


Alpine : 


Matador : A term used to identify a foreign bond issued in Spain by a company that is not domiciled in Spain. Matador bonds were bonds denominated in pesetas, and were usually corporate bonds. The market for matador bonds grew rapidly between 1987 and 1999, and attracted many large local and foreign investors. The name matador originated from the bullfighters in Spain.

Bulldog : A type of bond purchased by buyers interested in earning a revenue stream from the British pound or sterling. A bulldog bond is traded in the United Kingdom. If the revenue is used to reduce debt also in British pounds, the exchange rate risk is decreased. These bonds are issued by non-British institutions that want to sell the bond in the United Kingdom. U.S. investors can also purchase this bond, but by doing so they take on the risk of the change in value of the sterling.


Samurai : A yen-denominated bond issued in Tokyo by a non-Japanese company and subject to Japanese regulations. Other types of yen-denominated bonds are Euroyens issued in countries other than Japan.


Yankee : A bond denominated in U.S. dollars that is publicly issued in the U.S. by foreign banks and corporations. According to the Securities Act of 1933, these bonds must first be registered with the Securities and Exchange Commission (SEC) before they can be sold. Yankee bonds are often issued in tranches and each offering can be as large as $1 billion.






--------------------------------------------------------------------------------


(o) Underwriting  : The process by which investment bankers raise investment capital from investors on behalf of corporations and governments that are issuing securities (both equity and debt). 


 (p) Underwriter : A company or other entity that administers the public issuance and distribution of securities from a corporation or other issuing body. An underwriter works closely with the issuing body to determine the offering price of the securities, buys them from the issuer and sells them to investors via the underwriter's distribution network.


Underwriters generally receive underwriting fees from their issuing clients, but they also usually earn profits when selling the underwritten shares to investors. 

However, underwriters assume the responsibility of distributing a securities issue to the public. If they can't sell all of the securities at the specified offering price, they may be forced to sell the securities for less than they paid for them, or retain the securities themselves.


Book Runner : The main underwriter or lead manager in the issuance of new equity, debt or securities instruments. 

In investment banking, the book runner is the underwriting firm that "runs," or who is in charge, of the books. 


A large, leveraged buyout could involve multiple companies, and the book runner works with the other participating firms. Typically, one company takes the responsibility of "running" or handling the books, and the book runner is listed first among the other underwriters participating in the issuance. More than one book runner can manage a security issuance, in which case the involved parties are called "joint book runners."




--------------------------------------------------------------------------------


(r) Bought Deal : A new share issue that is bought entirely by one underwriter to resell to investors.


An underwriter will only do a bought deal if it is confident there is enough demand for the shares.



(u) Default rate: The rate at which debt holders default on the amount of money that they owe. 

It is often used by credit card companies when setting interest rates, but also refers to the rate at which corporations default on their loans. 

Default rates tend to rise during economic downturns, since investors and businesses see a decline in income and sales while still required to pay off the same amount of debt.




--------------------------------------------------------------------------------



1.C.3.1: The Bond Market

introduction

financing requirements of governments

euro dollars markets


--------------------------------------------------------------------------------


I.C.3.2 The Players

investment banks

institutional investors

market professionals

proprietary traders

market makers


--------------------------------------------------------------------------------


1.C.3.3. Bond by Issuers

government bonds

municipal bonds

foreign bonds

eurobonds

US Agency bonds

    TVA fannie mae sallie mae

Sinking fund

Release of Property

The markets


the government bond market

repo market



US Agency Bonds
These are issued by different organizations, seven of which dominate the US Market in terms of outstanding debt.

\begin{enumerate}
\item Federal National Mortgage Corporation (Fannie Mae)
\item Federal Home Loan Bank System (FHLBS)
\item Federal Home Loan Mortgage Corporation (Freddie Mac)
\item Farm Credit System (FCS)
\item Student Loan Marketting Association (Sallie Mae)
\item Resolution Finding Corporation (Refcorp)
\item Tennessee Valley Authority (TVA)
\end{enumerate}

%--------------------%

Agencies have common features

\begin{itemize}
\item They were created to fulfil a public purpose.
\item The debt is not necessarily guaranteed by the government
\end{itemize}

Municipal Securities

Local Government raise funds by issuing securities. Typically bonds issued in this sector are exempt from 
federal income tax. Hence the sector is known as the Tax Exempt Sector.

There are two generic types of municipal bond
\begin{itemize}
\item General Obligation Bonds
\item Revenue Bonds
\end{itemize}
Revenue bonds are secured by revenues generated by the projects these bonds funded.

