PRMIA 1.C.6 Credit Derivatives 

PRMIA 1.C.6 Credit Derivatives
1.B.6.1 Introduction
1.C.6.2. Credit Default Swaps
1.B.6.3 Credit Linked Notes
1.B.6.6 Collateralized Debt Obligations
1.B.6.8 Unintended Risks in Credit Risks

1) Introduction
2) Credit Default Swaps (CDSs)
3) Credit Linked Noted
4) Total Return Swaps
5) Credit Options
6) Synthetic CDOs
7) General Applications of credit derivatives
8) Unintended Risks in Credit Derivatives
9) Summary

1.B.6.1 Introduction
A technical Default is a delay in timely payment of an obligation, or a non-payment altogether.
If an obligor misses a payment, by even one day, it is said to be in techincal default.


1.C.6.2. Credit Default Swaps 
A credit derivative is an OTC derivative designed to transfer credit risk from one party to another. By synthetically creating or eliminating credit exposures, they allow institutions to more effectively manage credit risks. Credit derivatives take many forms. Three basic structures include:

credit default swap: Two parties enter into an agreement whereby one party pays the other a fixed periodic coupon for the specified life of the agreement. The other party makes no payments unless a specified credit event occurs. Credit events are typically defined to include a material default, bankruptcy or debt restructuring for a specified reference asset. If such a credit event occurs, the party makes a payment to the first party, and the swap then terminates. The size of the payment is usually linked to the decline in the reference asset's market value following the credit event.

total return swap: Two parties enter an agreement whereby they swap periodic payment over the specified life of the agreement. One party makes payments based upon the total return—coupons plus capital gains or losses—of a specified reference asset. The other makes fixed or floating payments as with a vanilla interest rate swap. Both parties' payments are based upon the same notional amount. The reference asset can be almost any asset, index or basket of assets.
	
credit linked note: A debt instrument is bundled with an embedded credit derivative. In exchange for a higher yield on the note, investors accept exposure to a specified credit event. For example, a note might provide for principal repayment to be reduced below par in the event that a reference asset defaults prior to the maturity of the note.

The fundamental difference between a credit default swap and a total return swap is the fact that the credit default swap provides protection against specific credit events. The total return swap provides protection against loss of value irrespective of cause—a default, market sentiment causing credit spreads to widen, etc.

Most credit derivatives entail two sources of credit exposure: one from the reference asset and the other from possible default by the counterparty to the transaction.



 
1.B.6.3 Credit Linked Notes
Credit linked notes are essentially hybrid instruments that combine a credit derivative with a vanilla bond.
 
 
1.B.6.4 Total Return Swaps
This is an agreement between two parties to exchange the total returns from financial assets, designed to transfer the credit risk from one party to another.
 
The TRS is a swap agreement in which the total return of a bank loan or credit-sensitive security is exchanged for some other cash flow, usually tied to LIBOR or some other loand or credit-sensitive security.
1.B.4.6.1 Synthetic Repos
1.B.4.6.2 Reduction in Credit Risk
1.B.6.6 Collateralized Debt Obligations
Collateralized debt obligations are securitized interests in pools of—generally non-mortgage—assets. Assets—called collateral—usually comprise loans or debt instruments. A CDO may be called a collateralized loan obligation (CLO) or collateralized bond obligation (CBO) if it holds only loans or bonds, respectively. Investors bear the credit risk of the collateral. Multiple tranches of securities are issued by the CDO, offering investors various maturity and credit risk characteristics. Tranches are categorized as senior, mezzanine, and subordinated/equity, according to their degree of credit risk. 
 
If there are defaults or the CDO's collateral otherwise underperforms, scheduled payments to senior tranches take precedence over those of mezzanine tranches, and scheduled payments to mezzanine tranches take precedence over those to subordinated/equity tranches. Senior and mezzanine tranches are typically rated, with the former receiving ratings of A to AAA and the latter receiving ratings of B to BBB. The ratings reflect both the credit quality of underlying collateral as well as how much protection a given tranch is afforded by tranches that are subordinate to it.
 
A CDO has a sponsoring organization, which establishes a special purpose vehicle to hold collateral and issue securities. Sponsors can include banks, other financial institutions or investment managers, as described below. Expenses associated with running the special purpose vehicle are subtracted from cash flows to investors. Often, the sponsoring organization retains the most subordinate equity tranch of a CDO.
 
1.B.6.8 Unintended Risks in Credit Risks


\newpage


PRMIA 1.C.6 The Futures Markets

Learning Outcome Statement

The candidate should be able to:
a.
  Define a futures contacts

b.
  Discuss some of the reasons that futures markets exist

c.
  Define open-outcry, contact size, tick size, limit up, limit down, expanded limit, initial margin, maintenance margin, mark-to-market, daily settlement,delivery month, offsetting          transaction,volume and open interest

d.
  Discuss types of orders in futures markets

e.
  Discuss the importance of standardization in futures contracts

f.
  Discuss the role of the clearing house

g.
  Compare and contrast physical delivery and cash-settlement

h.
  Discuss the process of physical settlement

i.
  Define and discuss the various types of orders

j.
  Define flex option

k.
  Discuss the exercise of an option on a futures contract

l.
  Discuss the various participants in futures markets: hedgers, speculators,managed futures investors

m.
  Calculate initial margin and change in margin due to market movements

n.
  Define calendar spread and basis




--------------------------------------------------------------------------------
1) Introduction

2) History of Forward Based Derivatives and Futures Markets

3) Futures Contracts and Markets

4) Options on Futures

5) Futures Exchanges and Clearing Houses

6) Market Participants: Hedgers

7) Market Participants: Speculators

8) Market Participants: Managed Futures Investors

9) Summary and Conclusion





--------------------------------------------------------------------------------


(j) Flexible Exchange Option (FLEX) is an option, generally written by a clearing house, whose expiration date, strike price, and exercising style can be modified.


A flex option provides flexibility as it can be tailored to meet an investor's specific needs.






--------------------------------------------------------------------------------
forward based derivatives

agricultural industry

future contracts and markets

general characteristics of future contracts and markets

daily tradining limits

settlement of future contracts

buying a futures contract

types of orders 

market order

best efforts or worked order

good til cancelled

market as open

stop order

marker if touched

spread order

margin requirements

leverage

reading a futures contract

reading a futures quotation page

market participants  hedgers

marking to market and margin

locals

day traders

position traders

intramarket spreads

example heating oil

commodity product spread

managed futures investors

summary and conclusion


summary assets that allow usets to lock in prices on assets

that are to be delivered

