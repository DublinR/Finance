\section{PRMIA 1.B.3 Futures and Forwards }
The candidate should be able to:
\begin{itemize}
 \item	Compare and Contrast forward and futures contracts
\item	 Discuss some uses of stock index futures
\item	 Define index point and value of an index point
\item	 Describe index arbitrage and program trading
\item	 Calculate a minimum variance hedge ratio for a portfolio of stocks, using futures, given beta
\item	 Describe some risks in index hedging
\item	 Discuss “tailing the hedge”
\item	 Compare and contrast currency forwards and futures contracts
\item	 Define covered interest parity
\item	 Calculate a forward exchange rate
\item	 Calculate a hedge ratio using foreign exchange futures
\item	 Discuss the relative basis risks with commodity futures
\item	 Define forward rate agreement (FRA)
\item	 Discuss FRAs, their nomenclature, uses and settlement
\item	 Calculate T-bill and Eurodollar futures prices
\item	 Construct a hedge using Eurodollar or T-bill futures
\item	 Define the tick value of a Eurodollar or T-bill futures contract
\item	 Define cheapest-to-deliver and conversion factor
\item	 Compare and contrast T-Bond and Gilt futures contracts
\item	 Define the tick value of a T-Bond and Gilt futures contract
\item	 Construct a hedge using T-bond futures
\item	 Compare and contrast stack and strip hedges
\end{itemize}



1) Introduction
2) Stock Indices Futures
3) Currency Forwards and Futures
4) Commodities Futures 
5) Forward Rate Agreements
6) Short Term Interest Rate Futures
7) T-Bond Futures  
8) Stack and Strip Hedges
9) Conclusion

\section{Futures and Forwards}

Fundamentally, forward and futures contracts have the same function: both types of contracts allow people to buy or sell a specific type of asset at a specific time at a given price. 

However, it is in the specific details that these contracts differ. First of all, futures contracts are exchange-traded and, therefore, are standardized contracts. Forward contracts, on the other hand, are private agreements between two parties and are not as rigid in their stated terms and conditions. Because forward contracts are private agreements, there is always a chance that a party may default on its side of the agreement. Futures contracts have clearing houses that guarantee the transactions, which drastically lowers the probability of default to almost never. 

Secondly, the specific details concerning settlement and delivery are quite distinct. For forward contracts, settlement of the contract occurs at the end of the contract. Futures contracts are marked-to-market daily, which means that daily changes are settled day by day until the end of the contract. Furthermore, settlement for futures contracts can occur over a range of dates. Forward contracts, on the other hand, only possess one settlement date.  

Lastly, because futures contracts are quite frequently employed by speculators, who bet on the direction in which an asset's price will move, they are usually closed out prior to maturity and delivery usually never happens. On the other hand, forward contracts are mostly used by hedgers that want to eliminate the volatility of an asset's price, and delivery of the asset or cash settlement will usually take place.  

\subsection{1.B.3.4 Commodity Futures}

\[F=S(1+T)(1+rT) = S +\]

Convenience yield = F - (S+)

The convenience yield arises because the holder of a spot commodity has the added advantage
that the he or she can supply their customers. if the commodity goes into short supply

\subsection{1.B.3.5 Forward Rate Agreement}
a Forward Rate Agreement is a form of forward contract that allows one to 'lock in'
or hedge interest-rate risk over a time specific period starting in the future.

\end{document}
