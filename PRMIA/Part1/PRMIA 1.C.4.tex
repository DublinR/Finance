PRMIA 1.C.4 The Foreign Exchange Markets
Learning Outcome Statement
The candidate should be able to:
\begin{itemize}	
\item	Define an exchange rate
\item	  Describe the interbank market
\item	  Define decentralized, continuous, open bid and double-auction
\item	  Define direct and indirect-term quotations
\item	  Compare and contrast direct dealing, foreign exchange brokers and electronic systems
\item	  Define the trading terms “mine” and “yours”
\item	  Define the trading term “big figure”
\item	  Define a cross-rate and a cross-trade
\item	  Calculate a cross-rate given two exchange rates
\item	  Describe some economic factors that might affect exchange rates
\item	  Discuss central bank intervention
\item	  Discuss spot and forward markets
\item	  Define currency swap rate, forward premium and forward discount
\item	  Calculate the forward premium or discount
\item	  Define covered-interest arbitrage / interest rate parity
\item	Describe a typical foreign exchange operation
\item	  Define front, middle and back office
\end{itemize}	

%-----------------------------------------------------------------------------------------%
1) Introduction
2) The Interbank Market
3) Exchange Rate Quotations
4) Determinants of Foreign Exchange Rates
5) Spot and Forward Markets
6) 
7) Summary and Conclusion
%-----------------------------------------------------------------------------------------%
\subsection*{Determinants of Exchange Rates}
Numerous factors determine exchange rates, and all are related to the trading relationship between two countries. Remember, exchange rates are relative, and are expressed as a comparison of the currencies of two countries. The following are some of the principal determinants of the exchange rate between two countries. Note that these factors are in no particular order; like many aspects of economics, the relative importance of these factors is subject to much debate.

1. Differentials in Inflation
As a general rule, a country with a consistently lower inflation rate exhibits a rising currency value, as its purchasing power increases relative to other currencies. During the last half of the twentieth century, the countries with low inflation included Japan, Germany and Switzerland, while the U.S. and Canada achieved low inflation only later. Those countries with higher inflation typically see depreciation in their currency in relation to the currencies of their trading partners. This is also usually accompanied by higher interest rates. 

2. Differentials in Interest Rates
Interest rates, inflation and exchange rates are all highly correlated. By manipulating interest rates, central banks exert influence over both inflation and exchange rates, and changing interest rates impact inflation and currency values. Higher interest rates offer lenders in an economy a higher return relative to other countries. Therefore, higher interest rates attract foreign capital and cause the exchange rate to rise. The impact of higher interest rates is mitigated, however, if inflation in the country is much higher than in others, or if additional factors serve to drive the currency down. The opposite relationship exists for decreasing interest rates - that is, lower interest rates tend to decrease exchange rates. 

3. Current-Account Deficits
The current account is the balance of trade between a country and its trading partners, reflecting all payments between countries for goods, services, interest and dividends. A deficit in the current account shows the country is spending more on foreign trade than it is earning, and that it is borrowing capital from foreign sources to make up the deficit. In other words, the country requires more foreign currency than it receives through sales of exports, and it supplies more of its own currency than foreigners demand for its products. The excess demand for foreign currency lowers the country's exchange rate until domestic goods and services are cheap enough for foreigners, and foreign assets are too expensive to generate sales for domestic interests. 

4. Public Debt
Countries will engage in large-scale deficit financing to pay for public sector projects and governmental funding. While such activity stimulates the domestic economy, nations with large public deficits and debts are less attractive to foreign investors. The reason? A large debt encourages inflation, and if inflation is high, the debt will be serviced and ultimately paid off with cheaper real dollars in the future. 

In the worst case scenario, a government may print money to pay part of a large debt, but increasing the money supply inevitably causes inflation. Moreover, if a government is not able to service its deficit through domestic means (selling domestic bonds, increasing the money supply), then it must increase the supply of securities for sale to foreigners, thereby lowering their prices. Finally, a large debt may prove worrisome to foreigners if they believe the country risks defaulting on its obligations. Foreigners will be less willing to own securities denominated in that currency if the risk of default is great. For this reason, the country's debt rating (as determined by Moody's or Standard & Poor's, for example) is a crucial determinant of its exchange rate. 

5. Terms of Trade
A ratio comparing export prices to import prices, the terms of trade is related to current accounts and the balance of payments. If the price of a country's exports rises by a greater rate than that of its imports, its terms of trade have favorably improved. Increasing terms of trade shows greater demand for the country's exports. This, in turn, results in rising revenues from exports, which provides increased demand for the country's currency (and an increase in the currency's value). If the price of exports rises by a smaller rate than that of its imports, the currency's value will decrease in relation to its trading partners.

6. Political Stability and Economic Performance
Foreign investors inevitably seek out stable countries with strong economic performance in which to invest their capital. A country with such positive attributes will draw investment funds away from other countries perceived to have more political and economic risk. Political turmoil, for example, can cause a loss of confidence in a currency and a movement of capital to the currencies of more stable countries. 

%-----------------------------------------------------------------------------------------%

\subsection*{Foreign Exchange Markets 1.C.4}

Exchange rate is the price of one currency in terms of anpther. The largest component of the ForEx Market is the interbank market.

The InterBank market is an OTC market where large institutions will trade currency amongst themselves.

%-----------------------------------------------------------------------------------------%
% I.C.4.2 Interbank Market
\subsubsection*{The InterBank Market}


Described as a "decentralized, continuous, open bid double auction market"

\begin{description}
\item[Decentralized] OTC market without a single location. Most activities occur in three major centres : London , New York and Tokyo. To a lesser extent: Paris, Frankfurt and Toronto.

Decentralization means it is very hard to regulate. There is no regulator, although the Bank of International Settlements (BIS) will 
collect data.
 
\item[Continuous] Price change contrinuously

\item[Open Bid]
 
\end{description}

%-----------------------------------------------------------------------------------------% 
1C4
interbank market
decentralised
continuous
open bid
role of the usa dollar
primary vehicle currency for a century
market and quoting conventions
determinantsof foreign exchange
the fundamental approach
realGDP
inflation levels
forward contracts
options contracts
swaps
basis contacts

spot market and forward market
forward market
forward discounts and preniums
interest rate parity
covered interest arbritage example

 
 
 
