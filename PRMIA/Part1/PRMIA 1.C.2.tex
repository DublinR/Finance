
PRMIA 1.C.2 The Money Markets

  



--------------------------------------------------------------------------------


Learning Outcome Statement

The candidate should be able to:
•
  Describe the characteristics of fixed income instruments

•
  Define term, principal, interest rate and secured vs. unsecured

•
  Describe the types of deposits (demand, notice and fixed-term)

•
  Define a reference rate

•
  Describe a credit facility

•
  Discuss syndication

•
  Calculate the interest payment on a term repo

•
  Describe the Eurocurrency market, particularly the Eurodollar market

•
  Define “add-on” interest

•
  Define LIBOR

•
  Describe different types of money market securities

•
  Calculate the bond-equivalent yield of a T-bill

•
  Define a commercial paper and a promissory note




--------------------------------------------------------------------------------
 

1) Introduction

2) Characteristics of MMIs

3) Deposits and Loans

4) Money Market Securites

5) Summary


--------------------------------------------------------------------------------
1.C.2.1 Introduction 


Money Market Instruments 

The major purpose of financial markets is to transfer funds from lenders to borrowers. 

Financial market participants commonly distinguish between the "capital market" and the "money market".  

The money market  refer  to borrowing and lending for periods of a year or less.


Certificates of Deposit

A CD can be legally negotiable or nonnegotiable, depending on certain legal specifications of the CD. 

Negotiable CDs can be sold by depositors to other 

parties who can in turn resell them. Nonnegotiable CDs generally must be held by the depositor until maturity. 

During the late 1970s and early to mid-1980s, between 60 and 80 percent of large CDs issued by large banks were negotiable instruments. 

The Federal Reserve stopped collecting separate data on negotiable CDs in 1987.


Commercial Paper

Commercial paper is a short-term unsecured promissory note issued by corporations and foreign governments.  

It is a low-cost alternative to bank loans, for many large, credit worthy issuers.   

Issuers are able to efficiently raise large amounts of funds quickly and without expensive Securities and Exchange Commission (SEC) registration.




--------------------------------------------------------------------------------
1.C.2.2 Characteristics of Money Market Instruments

Term , Principal Interest Rate, Marketability, Security, Collateralization , Call or put futures

 


--------------------------------------------------------------------------------
1.C.2.3 Deposits and Loans


1) Deposits from Business

2) Loans to Business

3) Repurchase Agreements

4) International Markets

5) The London Interbank Offered Rate





--------------------------------------------------------------------------------
1.C.2.4

Money market securities are generally issued for one year or less.

 

1.C.2.4 Money Market Securities


1) Treasury Bills

2) Commercial Paper

3) Bankers Acceptances

4) Cerficates of Deposit




--------------------------------------------------------------------------------


1.C.2.5 Summary




--------------------------------------------------------------------------------




% PRMIA 1.C.2.1
Money markets provide lenders and borrowes with a great deal of flexibility in terms of borrowing and deposit.
As a result they are very active markets as well.

\subsubsection*{Characteristics of Money Market Instruments}
The Cash marker for interest rate assets abd liabilities can be thought of as two distinct but related markets

\begin{enumerate}
\item
\item
\end{enumerate}

%============================================%

% PRMIA 1.C.2.2

\begin{itemize}
\item Term
\item Principal
\item Interest Rate
\item MArketability
\item Security - Collaterallization (or lack thereof)
\item Put and Call Futures
\item Deposits and Loans
\end{itemize}

%============================================%
% PRMIA 1.C.2.3.1

\subsubsection*{Deposits from Businesses}

\begin{itemize}
\item Demand Deposits - no notice to bank, low interest payments
\item Notice Deposits - Savings accounts, maybe asked (but rarely) to provide noticce before withdrawal.
\item Fixed term Deposits -have fixed term where the deposit, with interest, must be paid to depositor on maturity date
\end{itemize}



%============================================%
% PRMIA 1.C.2.4


\subsection*{Money Market Securities}

Money Market Securities are loans that have been strucutured so that they can be traded among investors in the secondary market
with a wide variety of structures and characteristics.

Money Market Securities are initially issued with terms of one year or less. They allow investors to place their excess cash in short
term instruments that, all else being equal, are less risky than securities with longer terms.

They also allow investors to get a higher rate of return than they would from sitting in a traditional bank account, while at the same time providing the issuers
of money markets securities with a relatively low cost source of funding short term.

%============================================%
\subsubsection*{Risk}
\begin{itemize}
\item Interest Rate Risk
\item Credit Risk
\item Liquidity Risk
\end{itemize}

%============================================%
Money Markets are often traded in denominations that are too large for individual investors.
Cash management trusts and Money market mutual funds have become popular vehicles to enable small investors to participate in the markets.
