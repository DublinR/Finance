
\subsection*{Bivariate Data}

\begin{itemize}
\item Covariance
\item Covariance Matrix
\item Correlation Coefficient
\item Correlation Matrix
\end{itemize}

\subsection{Covariance}
In probability theory and statistics, covariance is a measure of how much two random variables change together. If the greater values of one variable mainly correspond with the greater values of the other variable, and the same holds for the smaller values, i.e., the variables tend to show similar behavior, the covariance is positive.[1] In the opposite case, when the greater values of one variable mainly correspond to the smaller values of the other, i.e., the variables tend to show opposite behavior, the covariance is negative. 
The sign of the covariance therefore shows the tendency in the linear relationship between the variables. 


%===============================%
Correlation Coefficient

\[ r_xy = \frac{\mathrm{cov}(X,Y)}{\mathrm{Var}(X)  \times \mathrm{Var}(Y)}

%===============================%
Correlation Matrix

\left[ 
1 & \rho_{xy} \\
\rho_{xy} & 1 \\ 
\right]


2.B.6.5 Case Study - Calculating the volatility of a portfolio



Volatility is a vry important paraemter in pricing financial options. Historical volatility is often used 
as a basis for forecasting volatility

Historical volatility is the annualized standard deviation of the continuously compounded returns to the underlying asset.

%---------------------------------------------------------------------%
