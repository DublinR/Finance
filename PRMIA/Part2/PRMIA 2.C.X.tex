% Section 2.C

PRMIA 2

Section 2.C Calculus

\begin{description}
\item[2.C.1] Differential Calculus
\item[2.C.2] Case Study - Modified Duration of a Bond
\item[2.C.3] Higher-Order Derivatives
\item[2.C.4] Financial Applications of a "nd Derivative
\item[2.C.5] Partial / Total differentiation
\item[2.C.6] Integral Calculus
\item[2.C.7] Optimisation
\end{description}

Section 2.D Linear Algebra
\begin{description}
\item[2.D.1] Matrix Alagebra
\item[2.D.2] Application to portfolio construction
\item[2.D.3] Quadratic form
\item[2.D.4] cholesky Decomposition
\item[2.D.5] Eigen value /  Eigen vectors
\end{description}



\begin{itemize}
\item[2.C.2] Modified Duration of a Bond
\item[2.C.4] Financial Applications of Second Derivatives 
\begin{itemize}
\item convexity, Convexity in aciton, the delta and gamma of an option
\end{itemize}
\end{itemize}
%================================================%

\subsection*{Present Value of a bond}

\[ PV = \frac{CF_1}{1+y} + \frac{CF_2}{(1+y)^2} + \frac{CF_3}{(1+y)^3} + \ldots \frac{CF_n}{(1+y)^n} \]

\subsection*{First Derivative w.r.t Yield}

\[ \frac{}{} =  \sum^n_{i=1} \left[ \frac{-i CF_i}{(1+y)^{i+1}}\right] \]

%================================================%

Example 2.C.8 Modified Duration

Consider a 3 year bond paying annual coupons of 4 and trading on a yield to maturity of 5\%.

The dirty bond price is given as

\[ PV =  \frac{4}{1.05} +  \frac{4}{1.05^2} + \frac{104}{1.05^3} = 97.2768 \]

The Modified Duration is therefore

\[ MD = \frac{}{} \left( \frac{\frac{4}{1.05}}{97.2768} + \frac{2\frac{4}{1.05^2}}{97.2768} +\frac{3\frac{4}{1.05^3}}{97.2768} \right) \]

= -2.7470

%==============================================%

Financial Analysts use the Modified Duration through the following Relationship

\begin{itemize}
\item \% Price Change for a basis point change in yield = \% \Del \P_b = MD \times 0.0001\]
\end{itemize}

% DEL  - Pyramid Symbol

%===========================================================%
%-----------------------------------------------------------------------------------------%

PRMIA 2C4

Delta and Gamma of an Option

The value of an option is a function of the price of the underlying asset (in addition to other factors).

The first derivative of the option value w.r.t. price is called the \textbf{delta}.

The second derivative of the option value w.r.t. price is called the \textbf{gamma}.

If W(s) is the value of a european call regarded as a function of $s$, the price of the underlying asset.

$\delta S$ small change in that price

\[ W(s + \delta S) \approx W(s) + delta \delta S + frac{1}{2}gamma \delta^2 S\]

%-----------------------------------------------------------------------------------------%

PRMIA 2C5

Differentiating Functions of Several Variables

Partial Differentiation

\[ \frac{\partial f}{\partial x} \frac{\partial f}{\partial y} \]

\textbf{Example}

\[ f(x,y) = x^2 +6xy +2y^3 \]

\begin{itemize}
\item w.r.t $x$
\[\frac{\partial f}{\partial x} = 2x +6y\]
\item w.r.t $x$
\[\frac{\partial f}{\partial y} = 6x +6y\]
\end{itemize}

Equations that use Partial differentials are known as P.D.Es.

Hessian Matrix
matrix of second partial derivatives with respect to each variable


%=========================================%
\subsection{Bond Convexity}
This is defined to be half of the second derivative of the bond price, divided by the bond price.
Convexity is measured is ``years squared".
\[ \mbox{Convexity} = \frac{\frac{d^2PV}{dy^2} }{2PV} \]

From Earlier example

The second derivative w.r.t yield is

\[ PV = \frac{4}{1.05^1} + \frac{4}{1.05^2} + \frac{104}{1.05^3}\]

\[ \frac{dPV}{dy} = \frac{-4}{1.05^2} + \frac{-2\times 4}{1.05^3} + \frac{-3\times 104}{1.05^4}\]

\[ \frac{d^2PV}{dy^2} = \frac{2 \times 4}{1.05^3} + \frac{ 6 \times 4}{1.05^4} + \frac{12\times 104}{1.05^5}\]

\[ \frac{d^2PV}{dy^2} = 1004.4962\]

with $PV = 97.277$ 
\[ \mbox{Convexity} = \frac{\frac{d^2PV}{dy^2} }{2PV} = \frac{1004.4962 }{2 \times 97.277 } = 5.163 \]



%=====================================================================%

Find the region under the curve between the given boundaries.
\LARGE
\[ \int^4_0 \sqrt{x} dx = \int^4_0 (x)^{0.5} dx \]

\[  \bigg \left[ \frac{(x)^{1.5}}{1.5} \right]^4_0 = \frac{2}{3} \left[ (x)^{1.5} \right]^4_0 \]


Answer = 16/3

%=====================================================================%

\[ \int (3x+1)^{20} dx  \]

\begin{itemize}
\item u = 3x+1
\item du/dx = 3
\end{itemize}

\[ \frac{1}{3} du = dx \]



\[ \int (3x+1)^{20} dx  = \int (u)^{20} \frac{1}{3} du  \]

answer

\[ \frac{(u)^{20}}{63} + c \]

%=====================================================================%
