PRMIA 4B Governance
 
These ten principles of corporate governance are based on common themes in a variety of literatures on corporate governance  
 
Principle One: Key Competencies
Principle Two: Resources and Processes
Principle Three: Ongoing Education and Development
Principle Four: Compensation Architecture
Principle Five: Independence of Key Parties
Principle Six: Risk Appetite
Principle Seven: External Validation
Principle Eight: Clear Accountability
Principle Nine: Disclosure and Transparency
Principle Ten: Trust, Honesty and Fairness of Key People
These principles are applied in the following areas: Board and Audit Committees, Risk Management Infrastructure
and Financial Accounting and Reporting Infrastructure and the Corporation as a Whole.
 

PRMIA 4B Governance
Key Competencies
Resources and Processes
Ongoing Education and Development
Compensation Architecture
Independence of Key Parties
Risk Appetite
External Validation
Clear Accountability
Disclosure and Transparency
Trust, honest and fairness of key people

Key Competencies 
The organisation should have at its disposal employees who have adequate knowledge, skills and expertise to perform the tasks assigned to them. These 
competencies can be gained through professional qualification or by experience in role. 

Resources and Processes 
Adequate levels of resource shall be in place to enable the organisation to operate effectively. The business and technological processes shall be fit for purpose. 

Ongoing Education and Development 
The organisation shall encourage all employees to keep abreast of the latest developments in their particular areas of expertise, through courses, conferences, 
journals and other education channels and shall make adequate resources available to enable this to occur. 

Compensation Architecture 
Employees should be remunerated adequately for the roles that they perform, where ‘adequately’ is defined using external references and benchmarks, and in a 
framework which is consistent with the type of risk-taking behaviour expected of the employee. 

Independence of Key Parties 
The organisation shall ensure that, at all times, key checks and balances are in place to assure effective governance. Functions such as Audit and Risk 
Management are to be independent and report directly to through senior management rather than through those for whom they serve as a check and balance. 
Risk Appetite 
The Board of Directors of an organisation shall determine, and officially record, its appetite for each category of risk within its published risk framework. It shall 
encourage a cascade of this approach throughout the whole of the organisation. Such appetite must be expressed in a measurable way that can inform decisions 
at lower levels in the organisation. 

External Validation 
All aspects of the governance framework of the organisation shall be periodically validated by an independent body or bodies, external to the organisation, to 
ensure that they are appropriate to the sector and  geographies in which the organisation operates and  consistent with the stated policies and public 
representations made by the organisation. 

Clear Accountability 
At all levels in the organisation, accountabilities should be clearly defined. Individuals should be clearly advised of their own accountabilities and of the 
consequences of not fulfilling them in a timely and appropriate manner. 

Disclosure and Transparency 
The Board of Directors and senior management shall adopt an approach of disclosure and transparency consistent with their stated policies and ensure that this 
approach is followed at all levels throughout the organisation. 

Trust, honest and fairness of key people 
The key people involved in the application of good governance and risk management must be trustworthy and honest and treat others fairly at all times. 
