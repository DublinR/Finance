
PRMIA III.B.3 Credit Exposures


PRMIA 3.B.3 Credit Exposures

III.B.3.3 Exposure Profiles

III.B.3.3.2 Exposures profiles of derivatives.

III.B.3.4 Mitigation of Exposures


III.B.3.1 Introduction


Sources of credit exposures
1.
Direct and fixed Exposures

2.
Commitment

3.
Variable Exposure




III.B.3.2 Pre-settlement versus Settlement Risk

1.
Pre-settlement Risk

2.
Settlement Risk



Pre-settlement Risk
•
Failure to perform on this or a related contract

•
Rating downgrade (usually to a class below investment grade)

•
Bankruptcy



Settlement Risk


III.B.3.3 Exposure Profiles


III.B.3.3.1 Exposures profiles of standard debt obligations.

Standard debt contracts usually have fairly straightforward exposure profiles.
•
Whenever a payment is received, the exposure drops by the payment amount.

•
Between exposure dates the exposure increases smoothly.



 

III.B.3.3.2 Exposures profiles of derivatives.

OTC derivatives contract such as swaps, forwards, and Foreign exchange transaction have special features that complicate the calculation of corresponding exposure amounts. Derivatives often have an initial value of zero (or close to zero). This means that the current exposure is a very bad measure of future Exposure, which can have dramatically with market movements.

These exposure profiles are time dependent and stochastic. At future dates, an O. T. C. derivative can have a positive or negative value, but the exposure is floored at zero. By definition, an exposure can not be negative. The exposure in the amount lost of the obligor defaults.


Quantile-based exposure measures

These measures are defined very much like VaR levels.

The p-quantile exposure at time T is the level below which the exposure will remain with probability p. It is denoted q*(p).

Pt [D(T) q*(p)] = p

E(t,T) = q*(p)


III.B.3.4 Mitigation of Exposures
1.
Netting agreements

2.
Collateral

3.
Other counter party risk mitigation procedures


