3.B.6 Credit Risk Capital Calculation

Learning Outcome Statement
The candidate should be able to:
 Explain the calculation of Economic Credit Capital using Credit
Portfolio Models
 Demonstrate Minimum Credit Capital Requirements under Basel I
 List the Weaknesses of the Basel I Accord for Credit Risk
 Explain the Latest proposal for Minimum Credit Capital requirements
 Describe the Standardised Approach in Basel II
 Describe the Internal Ratings Based Approach (IRB) for Corporate, Bank and
Sovereign Exposures
 Describe the Internal Ratings Based Approach (IRB) for Retail Exposures
 Describe the Internal Ratings Based Approach (IRB) for SME Exposures
 Describe the Internal Ratings Based Approach (IRB) for Specialised Lending
and Equity Exposures
 List the new components of Pillar II for credit risk
 Explain Credit Model Estimation and Validation in Basel II
 Describe Securitisation in Basel II
 Describe the application of credit risk contribution methodologies for
Economic Credit Capital Allocation
 Demonstrate the Shortcomings of VaR for Economic Credit Capital and
Coherent Risk Measures


1. Introduction
2. Economic Credit Capital Calculation
3. Regulatory Credit Capital Basel I
4. Regulatory Credit Capital Basel II
5. Basel II: Credit Model Estimation and Validation
6. Basel II: Securitisation
7. Advanced Topics on Economic Capital Credit
8. Summary and Conclusions

\newpage

PRMIA 3.B.6 Credit Risk Capital Calculation

PRMIA 3.B.6 Credit Risk Capital Calculation

3.B.6.1 Introduction

3.B.6.2 Economic Credit Capital Calculation

3.B.6.3 Regulatory Credit Capital : Basel I


%=======================================================================================%
3.B.6.1 Introduction

•
Credit Portfolio Models

•
The Basel Accord : Framework created by the Basel Comittee onn Banking Supervision (BCBS) that is now the basis for banking regulation across the world.

•
•
Basel II treatment for internal ratinf systems

•
Probability of default estimation

%=======================================================================================%

3.B.6.2 Economic Credit Capital Calculation

(3.B.6.2.1.1) Time Horizon

(3.B.6.2.1.2) Credit Loss Definition

(3.B.6.2.1.3) Quantile of the Loss Distribution


The time horizon and quantile are key parameters set by the management,

%=======================================================================================%

(3.B.6.2.2) Expected and Unexpected Losses

Credit reserves are traditionally set aside to absorb expected losses (EL) during the life of a transaction.

 

 

\[CreditVaR(L) =Q(L) - EL(\]


(3.B.6.2.3) Enterprise Credit Capital and Risk Aggregation

 

 %=======================================================================================%

 

Q(L) denotes the \% quantile of the portfolio loss distribution at the horizon.

 

 
%==========================================================================================%
\newpage
\subsection*{3.B.6.3 Regulatory Credit Capital : Basel I}

 

(3.B.6.3.1) Minimum Credit Capital Requirement under Basel I 

Step 1 : Credit Equivalent Assets

 

 

3.B.6.4 Regulatory Credit Capital : Basel II

 

%=======================================================================================%
 

%==========================================================%
\subsection*{3.B.6.5 Basel II:  Credit Model Estimation and Validation}


Methodologies for PD estimation

Point in Time and through the cycle Ratings

Minimum Standards by Quantification and Credit

Validation of estimates

\begin{itemize}
\item Benchmarking
\item Backtesting 
\end{itemize}

%==========================================================%
\subsection*{3.B.6.6 Basel II Securitisation}

traditional securitisation vs synthetic securitisation

regulatory arbitrage


 

\subsection*{3.B.6.7 Advanced Topics on Economic Credit Capital}

•
Application of credit risk contribution methodologies for ECC

•
The shortcomings of VaR for ECC and coherent risk measures




Sub-additivity is a property of risk measures required to account for portfolio diversificaiton.

 

\subsection*{3.B.6.8 Summary and Conclusion}

 


This chapter reviews the main concepts for estimating and allocating ECC, as well as the current regulatory framework for credit capital.


Credit portfolio models must be defined and parameterized consistently with the ECC definition of the firm.

This definition includes the time horizon, the type of credit loss (default only or mark to market) and the confidence level (or quantile) of the loss distribution.



Key credit risk components: PDs, Exposures, LGDs. 


 



%%--------------------------------------------------------------------------------%%
3.B.6.1. Introduction 


This chapter how credit portfolio models must be defined and parameterized consistently to measure economic credit capital._ from a bottom up approach.

 

The basic rules for computing minimum capital under the Basel accord:

Page 316.

\subsection*{Economic credit capital Calculations}

This Acts as a buffer that provides protection against credit risk (potential credit losses) faced by an institution

 

Traditionally I capital is designed to absorb unexpected losses up to a attain confidence interval. While Credit preserves are set aside to deal with expected losses

 

\subsection*{Credit portfolio model}

We must Consider the time horizon or holding period the definiteon of credit losses, the quantise (and definition) of unexpected losses. Covered by capital. The definitions of these three parameters are tightly linked to the actual definition by capital used by the firms.




%--------------------------------------------------------------------------------%


Page 317

Credit loss definitions

Default only Credit losses

Estimates of the probability of default P.D. loss given default L.G.D. and exposures at Default E. A. D.




