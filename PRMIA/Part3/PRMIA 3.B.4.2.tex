
3.B.4 Default and Credit Migration



3.B.4 Default and Credit Migration

3.B.4.1. Default Probabilities and Term Structures of Default Rates

3.B.4.2 Credit Ratings

3.B.4.2.1  Measuring rating accuracy

3.B.4.3 Agency Ratings

3.B.4.3.2 Transition Matrices, Default probabilities and Credit Migration

3.B.4.4 Credit Scoring and Internal Rating models

3.B.4.4.1 Credit Scoring

3.B.4.4.2 Estimation of the Probability of Default

3.B.4.4.1 Other methods to determine the Probability of Default.

3.B.4.5 Market Implied Default Probabilities

3.B.4.5.1 Pricing the Calibration Security

3.B.4.5.2 Calculating implied default probabilities

 

3.B.4.1. Default Probabilities and Term Structures of Default Rates

Default probability p1

Survival probability  1-p1

Odds of Default H1={p1}{1-p1}



--------------------------------------------------------------------------------


default probability per unit time(T)

This is also known as the default rate or default intensity or default hazard rate.

It is the “instantaneous” probability of default in a continuous time setting.


Survival Probability P(0,T) = exp {-\int (t) dt }

The probability of a default between times 0 and T is 1-P(0,T).








3.B.4.2 Credit Ratings

3.B.4.2.1  Measuring rating accuracy

 


 

An accuracy ratio of close to one means that the model is almost as good as the "ideal" model, while an accuracy ratio of 0 means that it is as bad as a purely random classification.


 



--------------------------------------------------------------------------------


3.B.4.3 Agency Ratings

Examples of statistical models are KMV’s Expected Default Frequencies

and Kamakura’s default probability estimates.


3.B.4.3.2 Transition Matrices, Default probabilities and Credit Migration

Calculation of Transition Probabilities
•
Time invariance

•
Markov Property




 

3.B.4.4 Credit Scoring and Internal Rating models

Variables
•
Balance sheet data capturing the indebtedness of the obligor.

•
Profits and free cash flows capturing the ability to pay.

•
The riskiness (volatility) of the business

•



3.B.4.4.1 Credit Scoring

One of the earliest published credit scoring model goes back to Altman (1968), and was further developed in Altman et all (1977).

Credit Scoring models usually rely on accounting ratios.

 

3.B.4.4.2 Estimation of the Probability of Default

The two most common models for estimation of default probabilities are logit and probit models.

 

3.B.4.4.3 Other methods to determine the Probability of Default.

The KMV model can be viewed as a scoring model where the accounting ratio is the distance to default.

This contains "market information" and "volatility information", which are not usually found in classical accouting ratios.


Neural Networks and Expert Systems

These systems have been found to be as effective as established statistical models such as logit and probit methods.

Adoption has been hindered by the fact that they are considered "black box" approaches.


 

3.B.4.5 Market Implied Default Probabilities

The credit default tree is a perfectly adequate model for pricing simple credit sensitive instruments such as corporate bonds or CDSs, provided that the necessary conditional default probabilities are already specified.

Instead of determining prices for given set of parameters, we know determined parameters for a given set of prices of a set of benchmark securities, the calibration securities.

The default probabilities reached in this calibration exercise are known as market-implied default probabilities.


3.B.4.5.1 Pricing the Calibration Security

 

 

E(0,T)

 

The value of defaultable coupon bond with coupon payments Tk  k=1,......K, coupon amount c and recovery rate R can be written as

 


 











 

 

3.B.4.5.2 Calculating implied default probabilities

 

3.B.4.6 Credit Rating and Credit Spreads

3.B.4.7 Summary 

 



\section{Accuracy Ratio} % PRMIA 3.B.4.2
%-------------------------------------------------------------%
This is the area ration between CAP of the credit rsk model under construction and the random models CAP (the diagonal), with the area
between perfect prediction models CAP (crystal ball) and the diagonal.

Here the x-axis is now also a percentage scale.

The Area between the ideal model and random model is easily found to be $1-D/2$, where D is the fraction of default obligators.

Thus the accuracy of any given model with CAP function $CAP(x)$

\[ AR = \int^{1}_{0} \frac{CAP(x)dx - 1/2}{1/2(1-D)}\]

\begin{itemize}
\item $AR \approx 0$ very bad
\item $AR \approx 1$ ideal model
\end{itemize}

%-------------------------------------------------------------%

Negative Accuracy is also possible - model even worse than purely random credit ranking and the model ranking shoule be inverted.

