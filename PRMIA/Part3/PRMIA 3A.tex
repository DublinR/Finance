\documentclass[]{report}


% Title Page
\title{PRMIA 3A}


\begin{document}
\maketitle

\section{Section 3A}
\subsection{PRMIA 3.0 Capital Allocation and RAPM}

\subsubsection{Learning Outcome Statement}
The candidate should be able to:
\begin{verbatim}
  Describe the Role of Capital in a Financial Institution
  Define and Describe the different types of capital
  Demonstrate Economic Capital
  Describe the different approaches to calculating Economic Capital
  Describe Regulatory Capital
  Explain the Basel Norms
  Explain the Derivation of Regulatory Capital
  Explain Capital Allocation
  Demonstrate the Risk Contribution Methodologies for Economic Capital Allocation
  Explain Risk Adjusted Performance Measurement (RAPM)
  Demonstrate Risk Adjusted Return On Capital (RAROC)
\end{verbatim}

\begin{enumerate}
\item Introduction
\item Economic Capital
\item Regulatory Capital
\item Capital Allocation and Risk Contribution
\item RAROC and Risk adjusted performance
\item Summary and conclusion
\end{enumerate}

Risk adjusted return on capital (RAROC) is a risk-based profitability measurement framework for analysing risk-adjusted financial performance and providing a consistent view of profitability across businesses. 

Risk-adjusted return on capital (RAROC) gives decision makers the ability to compare the returns on several different projects with varying risk levels.

RAROC was popularized by Bankers Trust in the 1980s as an adjustment to simple return on capital (ROC).

\subsection{3.0.1 Introduction}


\subsection{3.0.2 Economic Capital}
Economic capital is the amount of risk capital, assessed on a realistic basis, which a firm requires to cover the risks that it is running or collecting as a going concern, such as market risk, credit risk, and operational risk. It is the amount of money which is needed to secure survival in a worst case scenario.

Section 3 : The bottom-up approach to calculating EC
Section 4 : Stress testing of portfolio losses and Economic Capital
Section 5 : Entreprise Capital practices - Aggregation

\subsection{3.0.3 Regulatory Capital}
Regulatory capital is the mandatory capital the regulators require to be maintained by financial institutions.

Section 3. Basel I regulation.

Market Risk Capital

Cook ratio

Section 4.  Basel II accord

The Basel II accord consists of three pillars.
Minimum capital requirements
supervisory review
market discipline
\subsection{3.0.4 Capital Allocation and Risk Contribution}

1) Stand Alone EC contributions
2) Marginal EC contributions
3) Incremental EC contribution

Additive decomposition of EC is of the form
	EC =iECi

\subsection*{3.0.5 RAROC and Risk adjusted performance}
Section 1: Objectives of RAPM

RAPM: Risk adjuested performance measurement.

Section 2:  Mechanic of RAROC

Simplified General Formula

RAROC=revenues - costs - expected lossescapital

\subsection*{3.0.6 Summary and Conclusions}


Aside from ownership issues, the primary goal of cpaital in a firm is to act as a buffer against unexpected losses.

Three types of capital
1) actual physcial capital
2) Economic Capital
3) regulatory capital
%------------------------------------------------------------------------%


\subsection*{3.A.1 Introduction to Market Risk Management}

\subsection*{3.A.1.1 Introduction}

\subsection*{3.A.1.2 Market Risk}

Why is market risk management important?

Distinguishing MArket Risk from Other Risks?


\subsection{3.A.1.3 Market Risk Management Tools}

Steps
1) Identification
2) Assessment
3) Monitoring
4) Control

\subsection*{3.A.1.4 The Organisation of Market Risk Management}

4 Stylised facts
1) The Risk management fuction should be part of a framework, controlled by the board of directors
2) The Risk management function should operate indepedently
3) The Risk management function should produce regular reports of exposures
4) The Risk management process should be well-documented

\subsection{3.A.1.5 Market Risk Management in Fund Management}

1) Introduction
2) Risk Identification
3) Assessment
4) Control and Mitigation
 Selective Hedging
 Momentary hedging
 Managing for risk adjusted performance target
 Capital Protection

\subsection*{3.A.1.6 Market Risk Management in Banking}

1) Introduction
2) Risk Identification
3) Assessment
4) Control and Mitigation
 Delta Hedging


\subsection*{3.A.1.7 Market Risk Management in Non-Financial Firms}

1) Introduction
2) Risk Identification
3) Assessment
4) Control and Mitigation



\subsection{3.A.1.8 Summary of Chapter}


3.A.2.6 Historical Simulation VaR
1) The Basic Model
2) Weighted Historical Simulations
3) Advantages and Disadvantages of historical approaches

3.A.2.7 Mapping Positions to Risk Factors

Four basic building blocks
1) Spot Foreign Exchange Positions
2) Equity Positions
3) Zero-Coupon Bonds
4) future/forward positions

Mapping Spot Positions
Mapping Equity Posiitions
Mapping Zeo Coupon Bonds
Mapping Forwards/Futures Positions

(option VaR) Delta Gamma Positions

\subsection{3.A.2.8 Backtesting VaR Models}

Back-testing involves after-the-fact analysis of the performance of risk estimation models

3.A.2.9 Why Financial Markets Are Not "Normal"

Central limit Theorem - Law of Large Numbers

\subsection{3.A.2.10 Summary}

This Chapter introduces three basic VaR Models
Analytical
Historical Simulation
Monte Carlo Simulation

Portfolio returns are not normally distributed


\subsection*{3.A.1.7 Market Risk in non financial firms}
\textbf{ 3.A.1.7.2 Identification}\\
 The three risk management tasks are Idenification, assessment and control/mitigation. Identification is the most difficult of the three.
  
 This is because of the comparative lack of necessary financial expertise in these firms.
 Economic Risks
 3.A.1.7.3 Assessment
 Decision analysis methods: useful for making strategic choices in medium to long term.
 3.A.1.7.4 Mitigation and Control
\subsection{3.A.1.8 Summary of section}
  
 More to market risk than calculating "value at risk".
 Market risk is hidden in many places.
  
  
 
\subsection{PRMIA 3.A.2 Introduction to VaR models}
These days one of the major tasks of risk managers is to measure the risk using value-at-risk (VaR) models. The basic VaR models for market risk are covered in this section.

Introduction to Value Learning Outcome Statement at Risk Models
The candidate should be able to:
\begin{verbatim}
 Define Value-at-Risk VaR
 Discuss Internal Models for Market Risk Capital
 Demonstrate Analytical VaR Model
 Explain Monte Carlo Simulation VaR model
 Demonstrate Historical Simulation VaR model
 Describe Risk Factor Mapping
 Demonstrate Mapping Spot Positions
 Demonstrate Mapping Equity Positions
 Demonstrate Mapping Zero-Coupon Bonds
 Describe Mapping Forward/Futures Positions
 Demonstrate Mapping Complex Positions
 Demonstrate Mapping Options: Delta and Delta-Gamma Approaches
 Describe Backtesting of VaR models
 Explain Central Limit Theorem and non-normality of financial markets
\end{verbatim}


%PRMIA 3.A.2 
%Introduction to VaR models
%3.A.2.1 Introduction to Value at Risk Models.
%3.A.2.2. Definition of VaR
%3.A.2.4 Analytical VaR Models
%3.A.2.5 Monte Carlo Simulation VaR
%VaR Confidence and Time Horizons
%Sample Question 1
%Sample Question 2
%Sample Question 3: Cumulative accuracy plot

\subsection*{3.A.2.1 Introduction to Value at Risk Models}
limitations. correct interpretations.

\subsection{3.A.2.2. Definition of VaR}
Var is an estimate of the loss of a fixed set of trading positions that would be equalled or exceeded with a specified probability. 
 

It is never correct to consider VaR as a worst case scenario.
 
The use of VaR involves two arbitrarily chosen parameters - the holding period and the confidence level.

%------------------------------%
\subsection*{3.A.2.3 Internal Models for Market Risk Rating}
Framewwork set out by the Basel Accord

\textbf{\textit{square root of time}} rule
\subsection{3.A.2.4 Analytical VaR Models}

R is the  h-day returns are normally distributed with mean $\mu$ and standard deviation $\sigma$.

\textbf{\textit{Limitations}}:\\
1) Market value sensitivities often are not stables as the market conditions change.
2) Analytical VaR is particularly inappropriate if there are discontinuous payoffs in the portfolio.

\subsection{3.A.2.5 Monte Carlo Simulation VaR Methodology}

\begin{verbatim}
Stock price S, assumed to follo a geometrix brownian motion process
dss=dt +d

: expected (per unit time) rate of return

: is the spot of volatility of a stock price

d: Wiener process 

: is a drawing from the standard normal distribution

dt: drift term

(dt)1/2: random,term
\end{verbatim}

\subsection{3.A.2.7 VaR of Equity Portfolio}

\[VaR= - ZXmk=1nkxkX\]

Stock market portfolios of 5 stocks

The value of the portfolio is $\$1$ million

Assume equal investment in each stock xkX= 0.2

Assume daily holding period

Stock market beta 
B= 0.7	A= 0.9 C= 0.5 D= 0.3 E= 0.1 (sum 2.5)

Confidence intervals 95%

VaR= 1.6451,000,0000.0250.22.5
\begin{verbatim}
VaR= $20,561
\end{verbatim} 
VaR Confidence and Time Horizons
It is usual to set a very high confidence level when estimating VaR for capital requirements.
 
For limit setting for managing day to day positions, it is usual to set VaR confidence levels that are neither too low to be exceed too often nor too high as to be never exceeded.

The time horizon may be a horizon roughly corresponding to a period in which positions may be liquidated in an orderly day, which could be just one day in a highly liquid market, or a week or more for larger positions in illiquid market.
 
\textbf{Sample Question 1}\\
For a security with a daily standard deviation of 2\%, calculate the 10-day Var at the 95\% confidence interval. Assume expected daily return to be nil.

If the daily standard deviation is 10\%, the 10 day standard deviation is 0.0210= 0.063245. 
The value of Z at the 95\% confidence level is 1.64485.
 
Therefore the Var value is 1.644850.063245 = 0.1040 , i.e 10.4\%.
Sample Question 2
 
If an institution has $\$1000$ in assets , and $\$800$ in liabilities, what is the economic capital required to avoid insolvency at 99\% level of confidence. The VaR in respect of the assets
at 99\% confidence over a one year period is $\$100$.
 
The economic capital required to avoid insolvenct is just the asset VaR i.e. $\$100$. This means that if the worst case losses are realized, the institution would need to have a buffer equivalent to those losses which in the this case will be $\$100$, and this buffer is the economic capital.
 
The actual value of the liabilities is not relevant as they are considered "riskless" from the instition's point of view, i.e. they will be taken at full value. in this particular case, the institution has $\$200$ in capital which is more than the economic required.
 
\textbf{Sample Question 3:} Cumulative accuracy plot
A cumulative accuracy plot measures the accuracy of credit ratings assigned by rating agencies by considering the relative ranking of obligors according to the ratings given.
 
PRMIA 3.A.3 Advanced VaR Models 

\begin{verbatim}
Sections
1) Introduction
2) Standard Distributional Assumptions
3) Models of Volatility Clustering
4) Volatility Clustering and VaR
5) Alternative Solutions to Non-normality
6) Decomposition of VaR.
7) Principal Component Analysis
8) Conclusions
\end{verbatim}
%--------------------------------------------------%
\subsection*{3.A.3.4 Volatility Clustering}

Market shock is likely to be followed by a large return (in either direction) for some time.

By failing to take account of volatility clsutering a form would potentially take unduly large risks, or will hold insufficient capital, in periods of market crisis.

In addtion they would hold on to too much expensive capital at other times.
\begin{verbatim}
volatility can be incorporated into VaR using the "exponentially weighted moving average" (EWMA) approach.
GARCH models

Focuses ion volatility clustering
Addresses issues raised by heteroskedascity
All GARCH models share a positive correlation between risk yesterday and risk today (an "autoregressive" structure)

The simplest GARCH model consist of 2 equation which can be estimated together


conditional mean equations  ri= c +ei

conditional variance equations i=+i-12+i-12			>0, ,0

In the absence of a market shock, the variance will tend towards its steady state variance , defined by
\end{verbatim}


 
\subsection{3.A.4.5 Stress testing}

\begin{verbatim}
 choice of test
 	Regulatory requirements
 	Specific needs of users
 		Complexity of portfolio
 		Frequency of trade
 		Liquidity
 		Volatility
 		Strategies employed
 
 Types of Stress test
 historical
 hypothetical scenarios
 algorithmic
 
 Sector wide - proposed
 desk level
 portfolio level
 \end{verbatim}
 
\subsection{3.A.4.7: Hypothetical Scenarios}
 
 \begin{verbatim}
 3.A.4.7.1 Modifying the Covariance Matrix
 3.A.4.7.2 Specifying Factor Shocks ( to “create” an event)
 3.A.4.7.3 Systemic Events and Stress-Testing Liquidity
 3.A.4.7.4 Sensitivity Analysis
 3.A.4.7.5 Hybrid Models
 \end{verbatim}
 
 Kupiec (1998) proposed a methodology that os a particular hybrid of covariance matrix manipulation and economic scenarios.
 This approach can also be applied to hte the problem of missing historical data in specifying shocks that can be used in a historical scenario
 
%----------------------------------------------------%
\subsection{3.A.4.8 Algorithmic approaches to Stress testing}

Systematic approach to stress testing
The goal is to identify a search algorithm to identify the worst outcome for the portfolio within some defined feasible set

Factor push stress tests
This type of stress test is named because it involves "pushing" each individual market risk factor in the direction that results in a loss for the portfolio.

Factors
a push magntitude (M)
portfolio revaluation ( 1000 values computed, lower used)
each is repeated for each of the N market risks factors affecting the portfolio.

\textbf{Maximum Loss}://
a maximum loss scenario is defined as a set of changes in market risj factors that results in the losses, subject to some feasibility constraint on the allowable changes in market risk factors.
The constraint is necessary because the scenario requires some plausibility.

\subsection*{3.A.4.9. Extreme Value Theory as a Stress-tesing Methods}

EVT is based on limit laws which applies to extreme observations in a sample.
These laws allow parametric estimation of high quantiles of loss (negatie return) distributions without making any assumptios abut the shape of the return distrbution as a whole.  

Block Maxima
Peak over Threshold

 \subsection{3.A.5 Liquidity Risk Management }
 Learning Outcome Statement
 The candidate should be able to:
 \begin{verbatim}
   Describe the factors which determine liquidity risks, and their pricing considerations
   Identify the processes concerning collateral management
   Discuss the implications of managing liquidity across business lines, legal entities, and currencies
   List the elements of funding diversification and market access
   Contrast the choices for intra-day management of liquidity
   Identify and differentiate the early warning signs of compromised liquidity
   Describe the components required for the disclosure of liquidity risk
   Identify, and design, the requirements of Stress Testing and a liquidity buffer
   Characterize the basic elements of financial contracts, their corresponding liquidity, and the relevance of time
   Describe the essential components of market, and funding, liquidity risk
   Discuss the impacts of counterparty / credit risk on liquidity relative to speads, defaults, credit enhancements, and asset based enhancements
   Describe the impact of behaviour on liquidity with respect to drawings, repayments, prepayments and draw-downs
   Derive the impact of insurance risk on liquidity
   Demonstrate the purpose, and effect of liquidity gap reports, and Liquidity at Risk (LAR)
   Describe the components of the contents used for internal and external liquidity reporting
  \end{verbatim}
  
 Liquidity risk is financial risk due to uncertain liquidity. An institution might lose liquidity if its credit rating falls, it experiences sudden unexpected cash outflows, or some other event causes counterparties to avoid trading with or lending to the institution. A firm is also exposed to liquidity risk if markets on which it depends are subject to loss of liquidity.
 
 Liquidity risk tends to compound other risks. If a trading organization has a position in an illiquid asset, its limited ability to liquidate that position at short notice will compound its market risk. Suppose a firm has offsetting cash flows with two different counterparties on a given day. If the counterparty that owes it a payment defaults, the firm will have to raise cash from other sources to make its payment. Should it be unable to do so, it too we default. Here, liquidity risk is compounding credit risk.
 
 Obviously, a position can be hedged against market risk but still entail liquidity risk. This is true in the above credit risk example—the two payments are offsetting, so they entail credit risk but not market risk. Another example is the 1993 Metallgesellschaft Debacle. Futures were used to hedge an OTC obligation. It is debatable whether the hedge was effective from a market risk standpoint, but it was the liquidity crisis caused by staggering margin calls on the futures that forced Metallgesellschaft to unwind the positions.
 
  
 
    
 Accordingly, liquidity risk has to be managed in addition to market, credit and other risks. Because of its tendency to compound other risks, it is difficult or impossible to isolate liquidity risk. In all but the most simple of circumstances, comprehensive metrics of liquidity risk don't exist. Certain techniques of asset-liability management can be applied to assessing liquidity risk. A simple test for liquidity risk is to look at future net cash flows on a day-by-day basis. Any day that has a sizeable negative net cash flow is of concern. Such an analysis can be supplemented with stress testing. Look at net cash flows on a day-to-day basis assuming that an important counterparty defaults.
 
 Obviously, such analyses cannot take into account contingent cash flows, such as cash flows from derivatives or mortgage-backed securities. If an organization's cash flows are largely contingent, liquidity risk may be assessed using some form of scenario analysis. Construct multiple scenarios for market movements and defaults over a given period of time. Assess day-to-day cash flows under each scenario. Because balance sheets differed so significantly from one organization to the next, there is little standardization in how such analyses are implemented.
 
 Regulators are primarily concerned about systemic implications of liquidity risk.
 
 
 


\end{document}