



Liquidity Risk
Basles II capital adequacy accord
Section 4 : Credit Risk
Section 5: Balance sheet management, liquidity risk and interest rate risk
1. Asset and liability management (ALM)
Section 6 : Risk transfer: securitisation and credit derivatives
Section 7
Section 8






Delegated monitoring
Asset Diversification monitoring
The Free rider problem
Expected loss given defualt
Liquidity Risk
Basles II capital adequacy accord
Explain the nature of liquidity risk, and discuss the insights from the theory of bank runs.
diamond and dyvbig (1983)
 
Section 4 : Credit Risk
Rating systems 
Constituents of credit risk 
Credit risk management
Credit risk models


Section 5: Balance sheet management, liquidity risk and interest rate risk
Asset and liability management (ALM)
Issues associated with ALM
Liquidity gap analysis 
Interest rate gap analysis 
Interest-margin variance analysis (IMVA)


1. Asset and liability management (ALM)




In banking, asset and liability management is the practice of managing risks that arise due to mismatches between the assets and liabilities (debts and assets) of the bank. This can also be seen in insurance.
Banks face several risks such as the liquidity risk, interest rate risk, credit risk and operational risk. Asset Liability management (ALM) is a strategic management tool to manage interest rate risk and liquidity risk faced by banks, other financial services companies and corporations.


Banks manage the risks of Asset liability mismatch by matching the assets and liabilities according to the maturity pattern or the matching the duration, by hedging and by securitization. Much of the techniques for hedging stem from the delta hedging concepts introduced in the Black-Scholes model and in the work of Robert C. Merton and Robert A. Jarrow. The early origins of asset and liability management date to the high interest rate periods of 1975-6 and the late 1970s and early 1980s in the United States. Van Deventer, Imai and Mesler (2004), chapter 2, outline this history in detail.


Modern risk management now takes place from an integrated approach to enterprise risk management that reflects the fact that interest rate risk, credit risk, market risk, and liquidity risk are all interrelated. The Jarrow-Turnbull model is an example of a risk management methodology that integrates default and random interest rates. The earliest work in this regard was done by Robert C. Merton. Increasing integrated risk management is done on a full mark to market basis rather than the accounting basis that was at the heart of the first interest rate sensivity gap and duration calculations.


The Jarrow Turnbull Model is a credit pricing model that utilizes multi-factor and dynamic analysis of interest rates.
Jarrow and Turnbull created this model in the attempt to identify a pattern between interest rate fluctuations and the probability of default over a specified time period. 
The significance of this model is its usage in pricing credit-based vehicles.






Interest-margin variance analysis (IMVA)


Section 6 : Risk transfer: securitisation and credit derivatives
Section 7
Section 8
 
 
