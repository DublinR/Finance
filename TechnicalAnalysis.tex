\documentclass[]{article}

%opening
\title{Technical Analysis}

\begin{document}

\maketitle


\section{What is Technical Analysis?}

Technical Analysis is the forecasting of future financial price movements based on an examination of past price movements. Like weather forecasting, technical analysis does not result in absolute predictions about the future. 
\newline

Instead, technical analysis can help investors anticipate what is "likely" to happen to prices over time. Technical analysis uses a wide variety of charts that show price over time.
\newline

Technical analysis is applicable to stocks, indices, commodities, futures or any tradable instrument where the price is influenced by the forces of supply and demand. Price refers to any combination of the open, high, low, or close for a given security over a specific time frame. (hence the acronym OHLC).
\newline

The time frame can be based on intraday (1-minute, 5-minutes, 10-minutes, 15-minutes, 30-minutes or hourly), daily, weekly or monthly price data and last a few hours or many years. 

\begin{itemize} 
\item Technical Analysis is a method of evaluating securities by analyzing 
statistics generated by market activity, such as past prices and volume. 

\item Technical analysts do not attempt to measure a security's intrinsic value, but instead use charts and other tools to identify patterns that can suggest future activity.

\item Technical analysts believe that the historical performance of stocks and markets are indications of future performance. 
\end{itemize}

\section{Indicators}

Indicators are calculations based on the price and the volume of a security that measure such things as money flow, trends, volatility and momentum. Indicators are used as a secondary measure to the actual price movements and add additional information to the analysis of securities. 
\newline

Indicators are used in two main ways: to confirm price movement and the quality of chart patterns, and to form buy and sell signals. 

%\section{Example of an Indicator}

Below are descriptions of three of the many TA indicators.

\subsection{Average Directional Index} 
The average directional index (ADX) is a trend indicator that is used to measure the strength of a current trend. The indicator is seldom used to identify the direction of the current trend, but can identify the momentum behind trends. 

\subsection{Moving Average Convergence }
The moving average convergence divergence (MACD) is one of the most well known and used indicators in technical analysis. This indicator is comprised of two exponential moving averages, which help to measure *momentum* in the security. 
\newline

The MACD is simply the difference between these two moving averages plotted against a centerline. The centerline is the point at which the two moving averages are equal. Along with the MACD and the centerline, an exponential moving average of the MACD itself is plotted on the chart. The idea behind this momentum indicator is to measure short-term momentum compared to longer term momentum to help signal the current direction of momentum. 

\subsection{On-Balance Volume} 
The on-balance volume (OBV) indicator is a well-known technical indicator that reflect movements in volume. It is also one of the simplest volume indicators to compute and understand. 
\newline

The OBV is calculated by taking the total volume for the trading period and assigning it a positive or negative value depending on whether the price is up or down during the trading period. When price is up during the trading period, the volume is assigned a positive value, while a negative value is assigned when the price is down for the period. The positive or negative volume total for the period is then added to a total that is accumulated from the start of the measure. 



\end{document}
