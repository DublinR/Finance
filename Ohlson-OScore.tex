\section{Ohlson o-score}
The Ohlson O-Score for predicting bankruptcy is a multi-factor financial formula postulated in 1980 by Dr. James Ohlson of the New York University Stern Accounting Department as an alternative to the Altman Z-score for predicting financial distress.[1]

\subsection{Calculation of the O-Score}
The Ohlson O-Score is the result of a 9-factor linear combination of coefficient-weighted business ratios which are readily obtained or derived from the standard periodic financial disclosure statements provided by publicly traded corporations. Two of the factors utilized are widely considered to be dummies as their value and thus their impact upon the formula typically is 0.[2] When using an O-Score to evaluate the probability of company’s failure, then exp(O-Score) is divided by 1 + exp(O-score).[3]

The calculation for Ohlson’s O-Score appears below:[4]

\[{\displaystyle {\begin{aligned}T={}&-1.32-0.407\log(TA_{t}/GNP)+6.03{\frac {TL_{t}}{TA_{t}}}-1.43{\frac {WC_{t}}{TA_{t}}}+0.0757{\frac {CL_{t}}{CA_{t}}}\\[10pt]&{}-1.72X-2.37{\frac {NI_{t}}{TA_{t}}}-1.83{\frac {FFO_{t}}{TL_{t}}}+0.285Y-0.521{\frac {NI_{t}-NI_{t-1}}{|NI_{t}|+|NI_{t-1}|}}\end{aligned}}} {\displaystyle {\begin{aligned}T={}&-1.32-0.407\log(TA_{t}/GNP)+6.03{\frac {TL_{t}}{TA_{t}}}-1.43{\frac {WC_{t}}{TA_{t}}}+0.0757{\frac {CL_{t}}{CA_{t}}}\\[10pt]&{}-1.72X-2.37{\frac {NI_{t}}{TA_{t}}}-1.83{\frac {FFO_{t}}{TL_{t}}}+0.285Y-0.521{\frac {NI_{t}-NI_{t-1}}{|NI_{t}|+|NI_{t-1}|}}\end{aligned}}}\]
where

TA = total assets
GNP = Gross National Product price index level
TL = total liabilities
WC = working capital
CL = current liabilities
CA = current assets
X = 1 if TL > TA, 0 otherwise
NI = net income
FFO = funds from operations
Y = 1 if a net loss for the last two years, 0 otherwise
