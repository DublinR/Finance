The Sterling ratio (SR) is a measure of the risk-adjusted return of an investment portfolio.

While multiple definitions of the Sterling ratio exist, it measures return over average drawdown, versus the more commonly used max drawdown.[citation needed] While the max drawdown looks back over the entire period you’re analyzing and takes the worst point along that equity curve, a quick change of the look back allows one to see what the worst peak to valley loss was for each calendar year as well.[citation needed] From there, you average the drawdowns of each year to come up with an average annual drawdown. The original definition was most likely suggested by Deane Sterling Jones (a company no longer in existence):


{\displaystyle SR={\frac {CompoundROR}{ABS(Avg.AnnualDD-10\%)}}} SR=\frac{Compound ROR}{ABS(Avg.Annual DD -10\%)}


If you are putting in your drawdown as a negative number, then subtract the 10%, and then multiply the whole thing by a negative to result in a positive ratio. If putting the drawdown in as a positive number, then add 10% and your result is the same positive ratio.[citation needed]

To clarify the reason he (Deane Sterling Jones) used 10% in the denominator was to compare any investment with a return stream to a risk-free investment (T-Bills). He invented the ratio in 1981 when t-bills were yielding 10%. Since bills did not experience drawdowns (and a ratio of 1.0 at that time), he felt that any investment with a ratio greater than 1.0 had a better risk/reward tradeoff. The average drawdown was always averaged and entered as a positive number and then 10% was added to that value.[citation needed]

This version of the Sterling ratio may be adjusted to something more like a Sharpe ratio as follows:

{\displaystyle SR={\frac {Annual\ Portfolio\ Return-Annual\ Risk\operatorname {-} Free\ Rate}{Average\ Largest\ Drawdown}}} SR=\frac{Annual\ Portfolio\ Return - Annual\ Risk\operatorname{-}Free\ Rate}{Average\ Largest\ Drawdown}
