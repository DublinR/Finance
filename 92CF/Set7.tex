

Asymmetric information, agency costs and capital structure
Information asymmetry
Capital Structure Theory

Asymmetric information, agency costs and capital structure
Capital structure, governance problems and agency costs
Agency costs of outside equity and debt
Agency costs of free cash flows
Firm value and asymmetric information




Asymmetric information, agency costs and capital structure
Capital structure, governance problems and agency costs
Agency costs of outside equity and debt
Agency costs of free cash flows
Firm value and asymmetric information


Information asymmetry
Information asymmetry deals with the study of decisions in transactions where one party has more or better information than the other. This creates an imbalance of power in transactions which can sometimes cause the transactions to go awry. Examples of this problem are adverse selection and moral hazard. Most commonly, information asymmetries are studied in the context of principal-agent problems.


Adverse selection refers to a market process in which "bad" results occur when buyers and sellers have asymmetric information (i.e. access to different information): the "bad" products or services are more likely to be selected. A bank that sets one price for all its checking account customers runs the risk of being adversely selected against by its low-balance, high-activity (and hence least profitable) customers. Two ways to model adverse selection are with signaling games and screening games.

Capital Structure Theory
In finance, capital structure refers to the way a corporation finances its assets through some combination of equity, debt, or hybrid securities. A firm's capital structure is then the composition or 'structure' of its liabilities. For example, a firm that sells $20 billion in equity and $80 billion in debt is said to be 20% equity-financed and 80% debt-financed. The firm's ratio of debt to total financing, 80% in this example, is referred to as the firm's leverage. In reality, capital structure may be highly complex and include dozens of sources. 
Gearing Ratio is the proportion of the capital employed of the firm which come from outside of the business finance, e.g. by taking a short term loan etc.







