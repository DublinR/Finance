

Liquidity Risk
Basles II capital adequacy accord
Section 4 : Credit Risk
Section 5: Balance sheet management, liquidity risk and interest rate risk
1. Asset and liability management (ALM)
Section 6 : Risk transfer: securitisation and credit derivatives
Section 7
Section 8


Project evaluation: Hirschleifer analysis and Fisher separation; the NPV rule and IRR rules of investment appraisal; comparison of NPV and IRR; 'wrong' investment appraisal rules: payback and accounting rate of return.
Risk and return - the CAPM and APT: the mathematics of portfolios; mean-variance analysis; two-fund separation and the CAPM; Roll's critique of the CAPM; factor models; the arbitrage pricing theory.
Derivative assets - characteristics and pricing: definitions: forwards and futures; replication, arbitrage and pricing; a general approach to derivative pricing using binomial methods; options: characteristics and types; bounding and linking option prices; the Black-Scholes analysis.
Efficient markets - theory and empirical evidence: underpinning and definitions of market efficiency; weak-form tests: return predictability; the joint hypothesis problem; semi-strong form tests: the event study methodology and examples; strong form tests: tests for private information.
Capital structure: the Modigliani-Miller theorem: capital structure irrelevancy; taxation, bankruptcy costs and capital structure; the Miller equilibrium; asymmetric information - 1) the under-investment problem, asymmetric information; 2) the risk-shifting problem, asymmetric information; 3) free cash-flow arguments; 4) the pecking order theory; 5) debt overhang.
Dividend theory: the Modigliani-Miller and dividend irrelevancy; Lintner's fact about dividend policy; dividends, taxes and clienteles; asymmetric information and signalling through dividend policy.
Corporate governance: separation of ownership and control; management incentives; management shareholdings and firm value; corporate governance.
Mergers and acquisitions: motivations for merger activity; calculating the gains and losses from merger/takeover; the free-rider problem and takeover activity.

Section 6 The choice of corporate capital structure
Basic features of debt and equity 
The Modigliani–Miller theorem 
Modigliani–Miller and Black–Scholes
Modigliani–Miller and corporate taxation
Modigliani–Miller with corporate and personal taxation





















