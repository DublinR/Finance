\documentclass[]{article}

%opening
\title{R Finance Packages}
\author{www.Stats-Lab.com}

\begin{document}

\maketitle

%========================================================
\section{Quandl}

One of the limitations of data available from Yahoo and Google is that it only dates back to January of 2007, while fund tracking and analysis will often require a time series of 20 years or more of historical returns.  Furthermore, some of the index-tracking ETF's didn't come into existence until the early 2000's, and new tracking funds for emerging markets and other indices have even shorter histories.  

To get return data going back farther in time, it is common to look to futures markets, and Quandl provides a rich set of historical futures prices.

\subsection{xts: eXtensible Time Series}

Provide for uniform handling of `R`'s different time-based data classes by extending `zoo`, maximizing native format information preservation and allowing for user level customization and extension, while simplifying cross-class interoperability.

The idea behind `xts` is to offer the user the ability to utilize a standard `zoo` object, while providing an mechanism to customize the object's meta-data, as well as create custom methods to handle the object in a manner required by the user.

\section{zoo: Representing Time Series Data as zoo objects}
The `ts` class is rather limited, especially for representing financial data that is not regularly spaced. For
example, the ts class cannot be used to represent daily financial data because such data are only
observed on business days. 

That is, a business day time clock generally runs from Monday to Friday skipping the weekends. So data are equally spaced in time within the week but the spacing between
Friday and Monday is different. 

This type of irregular spacing cannot be represented using the ts class.

A very flexible time series class is `zoo` (`Zeileis' ordered observations). The zoo class was designed to handle
time series data with an arbitrary ordered time index. This index could be a regularly spaced sequence of
dates, an irregularly spaced sequence of dates, or a numeric index. A zoo object essentially attaches
date information with data.

\section{PerformanceAnalytics}
\textbf{Econometric tools for performance and risk analysis}
This package is a collection of econometric functions for performance and risk analysis. This package aims to aid practitioners and researchers in utilizing the latest research in analysis of non-normal return streams. In general, it is most tested on return (rather than price) data on a regular scale, but most functions will work with irregular return data as well, and increasing numbers of functions will work with P$\&$L or price data where possible.

\section{TTR: Functions and data to construct technical trading rules with \texttt{R}.}

It is now possible to add dozens of technical analysis tools to chart very quickly to your analysis, with the `TTR` package..

The current indicators from the TTR package, as well as a few originating in the quantmod package are listed on the packages vignette document.

\end{document}
